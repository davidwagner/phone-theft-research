\documentclass{article}
\usepackage[top=1in,bottom=1in,left=1in,right=1in]{geometry}
\usepackage{amssymb}	% for \mathbb{}
\usepackage{enumerate}	% for \begin{enumerate}[(a)]
\usepackage{mathtools}
\usepackage{bytefield}
\usepackage{rotating}
\usepackage{listings}
\usepackage{subcaption}
\usepackage{caption}
\usepackage{graphicx}
\graphicspath{ {images/} }

\newpage
\title{Inferring Phone Behavior}
\author{TBD}
\date{TBD}
\begin{document}
\maketitle
\tableofcontents
\newpage

\section{Abstract}
\section{Introduction}

\section{Problem Structure \& Design}
\subsection{Problem}
TODO: Won
\subsection{Architecture}
The software can be divided into sections: 1) Determining where the phone is relative to the user using binary classifiers
2) A state-ful policy that combines the phone's location with its activity to track whether the phone should be activated or deactivated.

Before making any inference of a user's phone being in their possession, it was necessary to determine the location
of the phone relative to the user.
Classifiers were built to identify phone locations that were common throughout the phone's use and as disjoint from 
each other as possible. These phone locations were the phone being:
\begin{enumerate}
\item \textit{Table}|still on a table, or other flat surface
\item \textit{Pocket/Bag}|in the user's pocket or in a bag on their person
\item \textit{Steady State}|still in a bag, or other container in which the phone is not flat
\item \textit{Hand}|in the user's hand
\end{enumerate}

An individual classifier was built for each of these possible phone locations, due to the varying sets of features among
the locations. Each classifier was a binary classifier that indicated positively if the phone was in the corresponding location,
and negatively otherwise. Features were extracted from the sensor data naturally available to Android smartphones. The construction
and reasoning of each classifier is explained in detail in the next section.

In order to determine the phone's location based on the classifiers, relevant sensor data was divided into windows of one second. One second
was chosen because it was the maximum length of a window among the classifiers, so enough data for at least that amount of time was necessary
for all of the classifiers to output a classification. Since classifiers with smaller windows than this maximum length would then have multiple 
classifications within this window of one second (e.g. the table classifier would classify twenty of its 0.05s windows within one second), a classifier's
multiple classifications were then reduced to a single classification by taking the majority vote among the classifications. 

If multiple classifiers classified positively, that the phone was in the corresponding location, the phone's location was then determined by conservatively taking the most dangerous
of those locations. A location was considered dangerous if it was ambiguous if a phone would be with the user in that location (e.g. a phone still on the table
may be when the user has set it down momentarily or when the user has forgotten it and left it unattended on the table). Accordingly, a ranking of (most) danger
among the four phone locations was established as follows:
\begin{enumerate}
\item \textit{Table}
\item \textit{Steady State}
\item \textit{Pocket/Bag}
\item \textit{Hand}
\end{enumerate}

Finally, instead of directly assigning each second of time with its resulting classified location, each second of data was buffered in a "smoothing" window 
of 11 seconds in order to account for short, sudden changes in phone location that could not happen within the span of 1-2 seconds. For example, a 
"smoothing" window may contain five straight seconds of a \textit{Table} classification, then one second of a \textit{Pocket/Bag} classification, and then 
another five straight seconds of \textit{Table}. A phone would likely not be able to switch from those two locations within a single second, so the classification
of \textit{Pocket/Bag} was likely anomalous and that second should instead be classified as \textit{Table}. Therefore, the actual classification of a second of
data was classified as the majority classification among the five seconds before, the second itself, and the five seconds after. This introduced robustness
to fleeting outliers, although at the cost of an additional five seconds of latency. 

Following these preprocessing steps, the software would then have a concluded location of the phone at every second.

The software then combined the phone's location with phone unlock events in order to continuously decide whether the phone was in
the user's possession (activated state) or it was unable to be determined (deactivated state).


\subsection{Classifiers}
\subsubsection{Table}
TODO: Won
\subsubsection{Pocket/Bag}
TODO: Won
\subsubsection{Steady State}
TODO: Won
\subsubsection{Hand}
We wish to also identify when the phone is in the hand of the user.
 Assuming that the phone is passcode protected, this phone state is benign and implies that the user has control over the phone. 
The most common scenarios for this state is when the user is either actively using the phone or holding it [with straight arms pointing to the ground]
 Two telling characteristics of these two scenarios is the physical position of the phone and the user?s activity on the phone. 
The phone is most commonly positioned to be either at an angle with the screen facing up or with the side of the phone pointing down. 
To capture the physical position of the phone, we used the averages and standard deviations of the  acceleration and its magnitude of the x, y, z direction as features. 
The user could also be doing a range of actions, such as walking, with the phone in his hand, which could make the acceleration averages negligible. 
[In order to capture other movements of the user], we also featurized the number of times the acceleration changed signs. 

A typical scenario of a user actively using his phone is unlocking the keyguard of his phone and touching the screen multiple times and during which, the phone screen is on. 
To capture the active usage of the user, we used the number of times the user touch the screen and the fraction of time when the screen is one and when the phone is unlocked. 

We used soft margin SVM linear classifier with a RBF (Gaussian) kernel to classify between positives and negatives. 
\section{Evaluation}
\subsection{Diary Study}
TODO: Joanna
\subsubsection{Methodology}
\subsubsection{Results}

\subsection{Field Study}
TODO: Steven
\subsubsection{Methodology}
\subsubsection{Results}

\section{Related Works}
\section{Conclusion}



\end{document}