\section{Results}
\subsection{Single Phone Model}
Using the architecture detailed in Section 3, we trained the network for 10 epochs
and measured its effectiveness using $k$-fold cross-validation with $k = 10$.
Training/validation data was compiled from the four different diary studies
that used phone model WON.

To ensure that each fold had comparable training and validation distributions 
for each class, we preprocessed all of the diary study data before partitioning it
for cross-validation. Specifically, we divided the data of each diary study into its 
contiguous segments (e.g. 10:00-11:15, Table or 13:55-14:30, Hand), grouped 
those segments by class, and then concatenated the segments together for each class. The result was four 
homogenous datasets, one of each class, such that for each fold, we could construct
the validation set by aggregating together $\frac{1}{k}$ of each class dataset and then
construct the training set by compiling the remaining $\frac{k - 1}{k}$ of each class dataset.
Within these homogenous class datasets, the data were not randomized but 
kept sequential.

The average accuracy across all folds was $92.06\%$, and the corresponding
confusion matrix is shown in Figure X. The network is able to learn how to 
classify the Table state very accurately, which is understandable given 
the consistent nature of the state (i.e. always in a still, flat position). The states
of Backpack and Pocket were also classified with pretty moderate accuracy,
with the misclassifications likely stemming from the more diverse nature of 
the two states (i.e. the user could have been moving or still when the phone was
in their backpack or pocket). The Hand state was classified with the least accuracy,
and misclassified most often instead as the Backpack state. These misclassifications
likely also stem from the varied positions of the Hand state, as in addition to the user
using the phone actively in their Hand, they may also walk with their phone inactive
in their hand [@Joanna what kind of hand states did you have?]

 \begin{figure}[t]
 \center
\begin{tabular}{| l || c | c | c | c |}  
\toprule
\multicolumn{5}{c}{\textit{Actual Class}} \\ \cmidrule{2-5}
\textit{Predicted Class}		&	Backpack    & 	Pocket 	& 	Hand	&	Table \\
\midrule
Backpack			&	0.878 	&	0.108	&	0.368 	&	0.028 \\
Pocket			&	0.055 	&	0.843 	&	0.024 	&	0.001 \\
Hand			&	0.048 	&	0.016 	&	0.574 	&	0.006 \\
Table			&	0.019 	&	0.033 	&	0.035 	&	0.965\\
\bottomrule
\end{tabular}

\caption{Confusion matrix of the network predictions. Each entry indicates the proportion of
instances of the actual class that were predicted as the predicted class by the network.}
\centering
\label{fig:confusion}
\end{figure}  


\subsection{Multiple Phone Models}
We also attempted to validate our networks across different phone models, by training
on data collected by one phone model and then validating on data collected on a 
different phone model. We trained a network on all of the 
data collected from diary studies using phone model JOANNA, and then validated on 
two separate diary studies of data collected using phone model WON. In order to account
for inconsistent calibrations between the accelerometers of different phone models, we
attempted to recalibrated all accelerometer data using a linear calibration strategy. In detail,
using data from when the phone was flat on a table, we measured the acceleration in the
$X$, $Y$, and $Z$ directions and then computed the offsets from the expected values of
$0g$, $0g$, and $9.8g$. These offsets, or "calibration constants", were then added to 
all of the accelerometer data for that phone model.

Nonetheless, results between the two phones were inconclusive. On the first diary study with phone model WON, our network
had a validation accuracy of only $76\%$, but on the second diary study, our network had a validation
accuracy of $91\%$--an accuracy much more comparable to the cross validation accuracy
demonstrated in the single phone model case. These results suggest that a linear calibration
strategy may be effective, but further investigation is needed for definitive understanding.
