\section{Results}
\subsection{Single Phone Model}
Using the architecture detailed in Section 3, we trained the network for 10 epochs
and measured its effectiveness using $10$-fold cross-validation.
Training/validation data was compiled from the diary study conducted with the Nexus 5X.

The average accuracy across all folds was $92.06\%$, and the corresponding
confusion matrix is shown in Figure X. The network is able to learn how to 
classify the Table state very accurately, which is understandable given 
the consistent nature of the state (i.e. always in a still, flat position). The states
of Backpack and Pocket were also classified with pretty moderate accuracy,
with the misclassifications likely stemming from the more diverse nature of 
the two states (i.e. the user could have been moving or still when the phone was
in their backpack or pocket). The Hand state was classified with the least accuracy,
and misclassified most often instead as the Backpack state. These misclassifications
likely also stem from the varied positions of the Hand state, as in addition to the user
using the phone actively in their Hand, they may also walk with their phone inactive
in their hand [@Joanna what kind of hand states did you have?]


\begin{table}[h]
\caption{Confusion matrix of the network predictions. Each entry indicates the proportion of
total instances that were predicted as the predicted class by the network and labeled the actual class.}\label{fig:confusion} \centering
\begin{tabular}{| c || c | c | c | c }  
\toprule
\multicolumn{2}{c}{\textit{Class}}\multicolumn{3}{c}{(Actual)} \\ \cmidrule{1-5}
(Predicted)		&	Backpack    & 	Pocket 	& 	Hand	&	Table \\
\midrule
Backpack			&	0.266 	&	0.005	&	0.024 	&	0.029 \\
Pocket			&	0.016 	&	0.041 	&	0.001 	&	0.001 \\
Hand			&	0.017 	&	0.003 	&	0.035 	&	0.006 \\
Table			&	0.004 	&	0.002 	&	0.002 	&	0.965\\
\bottomrule
\end{tabular}
\end{table}


\subsection{Multiple Phone Models}
We also attempted to validate our networks across different phone models, by training
on data collected by one phone model and then validating on data collected on a 
different phone model. 

However, the same network trained on the Nexus 5X data could unfortunately not be immediately
applied to the Nexus 5 data, as 

We applied this calibration process to both the Nexus 5X and Nexus 5 data, and then
trained a network on all of the Nexus 5X data, and validated individually on each of the 
two days of diary study data collected from the Nexus 5.
Nonetheless, even after calibration, results between the two phones were inconclusive. 
On the first diary study with the Nexus 5, our network
had a validation accuracy of only $76\%$, but on the second diary study, our network had a validation
accuracy of $91\%$--an accuracy much more comparable to the cross validation accuracy
demonstrated in the single phone model case. These results suggest that a linear calibration
strategy may be effective, but further investigation is needed for definitive understanding.
