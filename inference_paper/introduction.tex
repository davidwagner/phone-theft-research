\section{Introduction}
With the recent proliferation of smartphones with sensors, such as light sensors and accelerometers,
 combined with the wide usage of smartphones over traditional phones, 
 there has been an explosion of applications utilizing sensors.
One such common application is detecting certain activites, when a person is walking or running, for example.

Knowing, a priori, the position of the phone (e.g. in a bag, in a user's pocket, etc.), 
 which we will call \textit{phone state}, can prove to be useful.
For example, a phone's vibration intensity can be adjusted based on where the phone is positioned on the person, potentially prolonging battery usage \cite{Fujinami}.
In another example, CO2 and pollution sensors could be turned off automatically if the phone were deemed to be in a pocket or backpack \cite{Miluzzo2010} 

Not only that, but phone state can also be a powerful information tool that can enhance the quality of other classifiers.
For example, work by Mart\'{\i}n et al has shown that feeding phone state as an input to a classifier predicting activity will increase accuracy in certain instances\cite{Martin}.  


In this paper, we discuss our methodology of automatically detecting and differentiating between different phone states: table, backpack, hand, and table. 
We trained a convolutional neural net on data collected from the accelerometer sensor 
 as well as screen events to achieve 92\% accuracy when evaluated on the same phone.
We then extend this to a different phone model and show that it is possible to evaluate states on a 
 different phone, even if the phone has an accelerometer sensor that is calibrated differently.
While the work on this is preliminary, we belive it provides an important step toward the goal of a 
universal classifier that can detect different phone states across phone models. 

