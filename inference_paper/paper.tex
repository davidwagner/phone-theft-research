\documentclass[a4paper,twoside]{article}
\usepackage{microtype}
\usepackage{amsmath}
\usepackage{amsfonts}
\usepackage{graphicx}
\usepackage{booktabs} % for professional tables
\usepackage{csquotes}

\usepackage{siunitx}
\usepackage{epsfig}
\usepackage{subfigure}
\usepackage{calc}
\usepackage{amssymb}
\usepackage{amstext}
\usepackage{amsmath}
\usepackage{amsthm}
\usepackage{multicol}
\usepackage{pslatex}
\usepackage{apalike}
\usepackage{SCITEPRESS}

\subfigtopskip=0pt
\subfigcapskip=0pt
\subfigbottomskip=0pt

\begin{document}

\title{Inferring Phone Location}
\author{\authorname{Steven Chen\sup{1}, Won Park \sup{1},  Joanna Yang and David Wagner \sup{2}}
\affiliation{\sup{1}Institute of Problem Solving, XYZ University, My Street, MyTown, MyCountry}
\affiliation{\sup{2}Department of Computing, Main University, MySecondTown, MyCountry}
\email{\{f\_author, s\_author\}@ips.xyz.edu, t\_author@dc.mu.edu}
}

\keywords{Sensors, Smartphone}

\abstract{Smartphone sensors are becoming more universal and more accurate. In this paper, we aim to distinguish between 4 positions a phone can be in: in the hand, pocket, backpack, or on a table. Using a convolutional neural network and data from the accelerometer and the screen state, we achieve a 92.04\% accuracy on the same phone. We also explore extending this to different phones. }
\onecolumn 


\maketitle \normalsize \vfill

\section{Introduction}
With the recent proliferation of smartphones with sensors, such as light sensors and accelerometers,
 combined with the wide usage of smartphones over traditional phones, 
 there has been an explosion of applications utilizing sensors.
One such common application is activity detection: detecting whether a person is walking or running, for example.

Knowing, a priori, the position of the phone can prove to be useful.
For exampo, a phone's vibration intensity can be adjusted based on where the phone is positined on the person, potentially prolonging battery usage \cite{Fujinami}.
In another example, CO2 and pollution sensors could be turned off automatically if the phone were deemed to be in a pocket or backpack \cite{Miluzzo2010} 

Not only that, but phone position can also be a powerful information tool that can enhance the quality of other classifiers.
For example, work by Mart\'{\i}n et al has shown that feeding phone position as an input to a classifier predicting activity will increase accuracy in certain instances\cite{Martin}.  


In this paper, we discuss our methodology of automatically detecting and differentiating between different phone states: table, backpack, hand, and table. 
We trained a convolutional neural net on data collected from the accelerometer sensor 
 as well as screen events to achieve 92.06\% accuracy when evaluated on the same phone.
We then extend this to a different phone model and show that it is possible to evaluate states on a 
 different phone, even if the phone has an accelerometer sensor that is calibrated differently.
While the work on this is preliminary, we belive it provides an important step toward the goal of a 
universal classifier that can detect different phone states across phone models. 


\section{Related work}
There has been much previous work involving smartphone sensors.
To note, this is in contrast to lots of previous work that involve sensors that are meant to be worn \cite{Kunze2005WhereAI}.
We focus on smartphone sensors since we believe they are more readily accesible, prevalent, and versatile than a wearable sensor.

Work by Khan et al, utilized accelerometer values to classify phone position 
but was limited to distinguishing between whether the phone was in the upper or lower half of the body.
Similarly, Miluzzo et al. proposed the use of various sensors to classify different phone locations, 
but only recognized two states: inside and outside of a pocket.


 In <TOFIX>'s work, the case of in-pocket detection was studied using light and proximity sensors[10, 3].
Several other works, including works by Fujinami et al. and by Coksun et al. tried to classify and distinguish between more states (e.g. hand, bag, pocket, etc.), but only used data from the accelerometer. 
In the former case, an accuracy of 74.6\% was achieved and in the latter case, an accuracy of roughly 84\% was achieved.

Park et al achieved an accuracy of 94\% in distinguishing between hand, ear, pocket, and backpack, but required that the participant be walking. 



Wiese et al explored the idea of adding sensors that utilize capacitive sensing,  multi-spectral properties, as well as light and proximity sensing.
They were able to achieve accuracies of 85\% to 100\%.
Similarly, Mart\'{\i}n et al used sensors like light, proximity, and acceleration sensors
to obtain position location accuracy of 92.94\%, but to cost in file size. 

In this paper, we take this idea further and utilize data from only two data sources: the accelerometer and the phone screen. 
Data from both sources are readily available, do not require specific user permissions, and are less energy draining than previously studied sensors.
Furthermore, unlike previous work, we also train and validate our model on phone data collected throughout the entire day, and not just during specific tasks (e.g. walking).
This includes times when the phone may be still and not actually directly on the user.








\section{Approach}

\subsection{Problem}
We want to predict the state of an user's phone based on the sensor data collected on the phone. 
From our observations and prior research, the most common states of the phone would be in a user's: backpack, pocket, hand, or on a table.
Upon a cursory observation of accelerometer traces, these phone states also appeared to be distinguishable and motivated
our approach to use deep learning to classify these states.
Sample accelerometer traces for the states, measured on a Nexus 5X
smartphone are shown in Figure ~\ref{fig:AccelDiffStates}.

\begin{figure}[h]
\begin{center}
 \scalebox{0.1}{%% Creator: Matplotlib, PGF backend
%%
%% To include the figure in your LaTeX document, write
%%   \input{<filename>.pgf}
%%
%% Make sure the required packages are loaded in your preamble
%%   \usepackage{pgf}
%%
%% Figures using additional raster images can only be included by \input if
%% they are in the same directory as the main LaTeX file. For loading figures
%% from other directories you can use the `import` package
%%   \usepackage{import}
%% and then include the figures with
%%   \import{<path to file>}{<filename>.pgf}
%%
%% Matplotlib used the following preamble
%%   \usepackage{fontspec}
%%   \setmainfont{Times New Roman}
%%   \setsansfont{Lucida Grande}
%%   \setmonofont{Andale Mono}
%%
\begingroup%
\makeatletter%
\begin{pgfpicture}%
\pgfpathrectangle{\pgfpointorigin}{\pgfqpoint{7.500000in}{5.000000in}}%
\pgfusepath{use as bounding box, clip}%
\begin{pgfscope}%
\pgfsetbuttcap%
\pgfsetmiterjoin%
\definecolor{currentfill}{rgb}{1.000000,1.000000,1.000000}%
\pgfsetfillcolor{currentfill}%
\pgfsetlinewidth{0.000000pt}%
\definecolor{currentstroke}{rgb}{1.000000,1.000000,1.000000}%
\pgfsetstrokecolor{currentstroke}%
\pgfsetdash{}{0pt}%
\pgfpathmoveto{\pgfqpoint{0.000000in}{0.000000in}}%
\pgfpathlineto{\pgfqpoint{7.500000in}{0.000000in}}%
\pgfpathlineto{\pgfqpoint{7.500000in}{5.000000in}}%
\pgfpathlineto{\pgfqpoint{0.000000in}{5.000000in}}%
\pgfpathclose%
\pgfusepath{fill}%
\end{pgfscope}%
\begin{pgfscope}%
\pgfsetbuttcap%
\pgfsetmiterjoin%
\definecolor{currentfill}{rgb}{1.000000,1.000000,1.000000}%
\pgfsetfillcolor{currentfill}%
\pgfsetlinewidth{0.000000pt}%
\definecolor{currentstroke}{rgb}{0.000000,0.000000,0.000000}%
\pgfsetstrokecolor{currentstroke}%
\pgfsetstrokeopacity{0.000000}%
\pgfsetdash{}{0pt}%
\pgfpathmoveto{\pgfqpoint{1.069861in}{0.809444in}}%
\pgfpathlineto{\pgfqpoint{7.205428in}{0.809444in}}%
\pgfpathlineto{\pgfqpoint{7.205428in}{4.468333in}}%
\pgfpathlineto{\pgfqpoint{1.069861in}{4.468333in}}%
\pgfpathclose%
\pgfusepath{fill}%
\end{pgfscope}%
\begin{pgfscope}%
\pgfsetbuttcap%
\pgfsetroundjoin%
\definecolor{currentfill}{rgb}{0.000000,0.000000,0.000000}%
\pgfsetfillcolor{currentfill}%
\pgfsetlinewidth{0.803000pt}%
\definecolor{currentstroke}{rgb}{0.000000,0.000000,0.000000}%
\pgfsetstrokecolor{currentstroke}%
\pgfsetdash{}{0pt}%
\pgfsys@defobject{currentmarker}{\pgfqpoint{0.000000in}{-0.048611in}}{\pgfqpoint{0.000000in}{0.000000in}}{%
\pgfpathmoveto{\pgfqpoint{0.000000in}{0.000000in}}%
\pgfpathlineto{\pgfqpoint{0.000000in}{-0.048611in}}%
\pgfusepath{stroke,fill}%
}%
\begin{pgfscope}%
\pgfsys@transformshift{1.348751in}{0.809444in}%
\pgfsys@useobject{currentmarker}{}%
\end{pgfscope}%
\end{pgfscope}%
\begin{pgfscope}%
\pgftext[x=1.348751in,y=0.712222in,,top]{\sffamily\fontsize{16.000000}{19.200000}\selectfont 0}%
\end{pgfscope}%
\begin{pgfscope}%
\pgfsetbuttcap%
\pgfsetroundjoin%
\definecolor{currentfill}{rgb}{0.000000,0.000000,0.000000}%
\pgfsetfillcolor{currentfill}%
\pgfsetlinewidth{0.803000pt}%
\definecolor{currentstroke}{rgb}{0.000000,0.000000,0.000000}%
\pgfsetstrokecolor{currentstroke}%
\pgfsetdash{}{0pt}%
\pgfsys@defobject{currentmarker}{\pgfqpoint{0.000000in}{-0.048611in}}{\pgfqpoint{0.000000in}{0.000000in}}{%
\pgfpathmoveto{\pgfqpoint{0.000000in}{0.000000in}}%
\pgfpathlineto{\pgfqpoint{0.000000in}{-0.048611in}}%
\pgfusepath{stroke,fill}%
}%
\begin{pgfscope}%
\pgfsys@transformshift{2.487075in}{0.809444in}%
\pgfsys@useobject{currentmarker}{}%
\end{pgfscope}%
\end{pgfscope}%
\begin{pgfscope}%
\pgftext[x=2.487075in,y=0.712222in,,top]{\sffamily\fontsize{16.000000}{19.200000}\selectfont 100}%
\end{pgfscope}%
\begin{pgfscope}%
\pgfsetbuttcap%
\pgfsetroundjoin%
\definecolor{currentfill}{rgb}{0.000000,0.000000,0.000000}%
\pgfsetfillcolor{currentfill}%
\pgfsetlinewidth{0.803000pt}%
\definecolor{currentstroke}{rgb}{0.000000,0.000000,0.000000}%
\pgfsetstrokecolor{currentstroke}%
\pgfsetdash{}{0pt}%
\pgfsys@defobject{currentmarker}{\pgfqpoint{0.000000in}{-0.048611in}}{\pgfqpoint{0.000000in}{0.000000in}}{%
\pgfpathmoveto{\pgfqpoint{0.000000in}{0.000000in}}%
\pgfpathlineto{\pgfqpoint{0.000000in}{-0.048611in}}%
\pgfusepath{stroke,fill}%
}%
\begin{pgfscope}%
\pgfsys@transformshift{3.625399in}{0.809444in}%
\pgfsys@useobject{currentmarker}{}%
\end{pgfscope}%
\end{pgfscope}%
\begin{pgfscope}%
\pgftext[x=3.625399in,y=0.712222in,,top]{\sffamily\fontsize{16.000000}{19.200000}\selectfont 200}%
\end{pgfscope}%
\begin{pgfscope}%
\pgfsetbuttcap%
\pgfsetroundjoin%
\definecolor{currentfill}{rgb}{0.000000,0.000000,0.000000}%
\pgfsetfillcolor{currentfill}%
\pgfsetlinewidth{0.803000pt}%
\definecolor{currentstroke}{rgb}{0.000000,0.000000,0.000000}%
\pgfsetstrokecolor{currentstroke}%
\pgfsetdash{}{0pt}%
\pgfsys@defobject{currentmarker}{\pgfqpoint{0.000000in}{-0.048611in}}{\pgfqpoint{0.000000in}{0.000000in}}{%
\pgfpathmoveto{\pgfqpoint{0.000000in}{0.000000in}}%
\pgfpathlineto{\pgfqpoint{0.000000in}{-0.048611in}}%
\pgfusepath{stroke,fill}%
}%
\begin{pgfscope}%
\pgfsys@transformshift{4.763723in}{0.809444in}%
\pgfsys@useobject{currentmarker}{}%
\end{pgfscope}%
\end{pgfscope}%
\begin{pgfscope}%
\pgftext[x=4.763723in,y=0.712222in,,top]{\sffamily\fontsize{16.000000}{19.200000}\selectfont 300}%
\end{pgfscope}%
\begin{pgfscope}%
\pgfsetbuttcap%
\pgfsetroundjoin%
\definecolor{currentfill}{rgb}{0.000000,0.000000,0.000000}%
\pgfsetfillcolor{currentfill}%
\pgfsetlinewidth{0.803000pt}%
\definecolor{currentstroke}{rgb}{0.000000,0.000000,0.000000}%
\pgfsetstrokecolor{currentstroke}%
\pgfsetdash{}{0pt}%
\pgfsys@defobject{currentmarker}{\pgfqpoint{0.000000in}{-0.048611in}}{\pgfqpoint{0.000000in}{0.000000in}}{%
\pgfpathmoveto{\pgfqpoint{0.000000in}{0.000000in}}%
\pgfpathlineto{\pgfqpoint{0.000000in}{-0.048611in}}%
\pgfusepath{stroke,fill}%
}%
\begin{pgfscope}%
\pgfsys@transformshift{5.902047in}{0.809444in}%
\pgfsys@useobject{currentmarker}{}%
\end{pgfscope}%
\end{pgfscope}%
\begin{pgfscope}%
\pgftext[x=5.902047in,y=0.712222in,,top]{\sffamily\fontsize{16.000000}{19.200000}\selectfont 400}%
\end{pgfscope}%
\begin{pgfscope}%
\pgfsetbuttcap%
\pgfsetroundjoin%
\definecolor{currentfill}{rgb}{0.000000,0.000000,0.000000}%
\pgfsetfillcolor{currentfill}%
\pgfsetlinewidth{0.803000pt}%
\definecolor{currentstroke}{rgb}{0.000000,0.000000,0.000000}%
\pgfsetstrokecolor{currentstroke}%
\pgfsetdash{}{0pt}%
\pgfsys@defobject{currentmarker}{\pgfqpoint{0.000000in}{-0.048611in}}{\pgfqpoint{0.000000in}{0.000000in}}{%
\pgfpathmoveto{\pgfqpoint{0.000000in}{0.000000in}}%
\pgfpathlineto{\pgfqpoint{0.000000in}{-0.048611in}}%
\pgfusepath{stroke,fill}%
}%
\begin{pgfscope}%
\pgfsys@transformshift{7.040371in}{0.809444in}%
\pgfsys@useobject{currentmarker}{}%
\end{pgfscope}%
\end{pgfscope}%
\begin{pgfscope}%
\pgftext[x=7.040371in,y=0.712222in,,top]{\sffamily\fontsize{16.000000}{19.200000}\selectfont 500}%
\end{pgfscope}%
\begin{pgfscope}%
\pgftext[x=4.137645in,y=0.442474in,,top]{\sffamily\fontsize{16.000000}{19.200000}\selectfont Time (ms)}%
\end{pgfscope}%
\begin{pgfscope}%
\pgfsetbuttcap%
\pgfsetroundjoin%
\definecolor{currentfill}{rgb}{0.000000,0.000000,0.000000}%
\pgfsetfillcolor{currentfill}%
\pgfsetlinewidth{0.803000pt}%
\definecolor{currentstroke}{rgb}{0.000000,0.000000,0.000000}%
\pgfsetstrokecolor{currentstroke}%
\pgfsetdash{}{0pt}%
\pgfsys@defobject{currentmarker}{\pgfqpoint{-0.048611in}{0.000000in}}{\pgfqpoint{0.000000in}{0.000000in}}{%
\pgfpathmoveto{\pgfqpoint{0.000000in}{0.000000in}}%
\pgfpathlineto{\pgfqpoint{-0.048611in}{0.000000in}}%
\pgfusepath{stroke,fill}%
}%
\begin{pgfscope}%
\pgfsys@transformshift{1.069861in}{1.034592in}%
\pgfsys@useobject{currentmarker}{}%
\end{pgfscope}%
\end{pgfscope}%
\begin{pgfscope}%
\pgftext[x=0.514848in,y=0.948926in,left,base]{\sffamily\fontsize{16.000000}{19.200000}\selectfont −10}%
\end{pgfscope}%
\begin{pgfscope}%
\pgfsetbuttcap%
\pgfsetroundjoin%
\definecolor{currentfill}{rgb}{0.000000,0.000000,0.000000}%
\pgfsetfillcolor{currentfill}%
\pgfsetlinewidth{0.803000pt}%
\definecolor{currentstroke}{rgb}{0.000000,0.000000,0.000000}%
\pgfsetstrokecolor{currentstroke}%
\pgfsetdash{}{0pt}%
\pgfsys@defobject{currentmarker}{\pgfqpoint{-0.048611in}{0.000000in}}{\pgfqpoint{0.000000in}{0.000000in}}{%
\pgfpathmoveto{\pgfqpoint{0.000000in}{0.000000in}}%
\pgfpathlineto{\pgfqpoint{-0.048611in}{0.000000in}}%
\pgfusepath{stroke,fill}%
}%
\begin{pgfscope}%
\pgfsys@transformshift{1.069861in}{1.872041in}%
\pgfsys@useobject{currentmarker}{}%
\end{pgfscope}%
\end{pgfscope}%
\begin{pgfscope}%
\pgftext[x=0.655365in,y=1.786375in,left,base]{\sffamily\fontsize{16.000000}{19.200000}\selectfont −5}%
\end{pgfscope}%
\begin{pgfscope}%
\pgfsetbuttcap%
\pgfsetroundjoin%
\definecolor{currentfill}{rgb}{0.000000,0.000000,0.000000}%
\pgfsetfillcolor{currentfill}%
\pgfsetlinewidth{0.803000pt}%
\definecolor{currentstroke}{rgb}{0.000000,0.000000,0.000000}%
\pgfsetstrokecolor{currentstroke}%
\pgfsetdash{}{0pt}%
\pgfsys@defobject{currentmarker}{\pgfqpoint{-0.048611in}{0.000000in}}{\pgfqpoint{0.000000in}{0.000000in}}{%
\pgfpathmoveto{\pgfqpoint{0.000000in}{0.000000in}}%
\pgfpathlineto{\pgfqpoint{-0.048611in}{0.000000in}}%
\pgfusepath{stroke,fill}%
}%
\begin{pgfscope}%
\pgfsys@transformshift{1.069861in}{2.709490in}%
\pgfsys@useobject{currentmarker}{}%
\end{pgfscope}%
\end{pgfscope}%
\begin{pgfscope}%
\pgftext[x=0.832122in,y=2.623823in,left,base]{\sffamily\fontsize{16.000000}{19.200000}\selectfont 0}%
\end{pgfscope}%
\begin{pgfscope}%
\pgfsetbuttcap%
\pgfsetroundjoin%
\definecolor{currentfill}{rgb}{0.000000,0.000000,0.000000}%
\pgfsetfillcolor{currentfill}%
\pgfsetlinewidth{0.803000pt}%
\definecolor{currentstroke}{rgb}{0.000000,0.000000,0.000000}%
\pgfsetstrokecolor{currentstroke}%
\pgfsetdash{}{0pt}%
\pgfsys@defobject{currentmarker}{\pgfqpoint{-0.048611in}{0.000000in}}{\pgfqpoint{0.000000in}{0.000000in}}{%
\pgfpathmoveto{\pgfqpoint{0.000000in}{0.000000in}}%
\pgfpathlineto{\pgfqpoint{-0.048611in}{0.000000in}}%
\pgfusepath{stroke,fill}%
}%
\begin{pgfscope}%
\pgfsys@transformshift{1.069861in}{3.546938in}%
\pgfsys@useobject{currentmarker}{}%
\end{pgfscope}%
\end{pgfscope}%
\begin{pgfscope}%
\pgftext[x=0.832122in,y=3.461272in,left,base]{\sffamily\fontsize{16.000000}{19.200000}\selectfont 5}%
\end{pgfscope}%
\begin{pgfscope}%
\pgfsetbuttcap%
\pgfsetroundjoin%
\definecolor{currentfill}{rgb}{0.000000,0.000000,0.000000}%
\pgfsetfillcolor{currentfill}%
\pgfsetlinewidth{0.803000pt}%
\definecolor{currentstroke}{rgb}{0.000000,0.000000,0.000000}%
\pgfsetstrokecolor{currentstroke}%
\pgfsetdash{}{0pt}%
\pgfsys@defobject{currentmarker}{\pgfqpoint{-0.048611in}{0.000000in}}{\pgfqpoint{0.000000in}{0.000000in}}{%
\pgfpathmoveto{\pgfqpoint{0.000000in}{0.000000in}}%
\pgfpathlineto{\pgfqpoint{-0.048611in}{0.000000in}}%
\pgfusepath{stroke,fill}%
}%
\begin{pgfscope}%
\pgfsys@transformshift{1.069861in}{4.384387in}%
\pgfsys@useobject{currentmarker}{}%
\end{pgfscope}%
\end{pgfscope}%
\begin{pgfscope}%
\pgftext[x=0.691606in,y=4.298721in,left,base]{\sffamily\fontsize{16.000000}{19.200000}\selectfont 10}%
\end{pgfscope}%
\begin{pgfscope}%
\pgftext[x=0.459292in,y=2.638889in,,bottom,rotate=90.000000]{\sffamily\fontsize{16.000000}{19.200000}\selectfont Acceleration (m/s\^2)}%
\end{pgfscope}%
\begin{pgfscope}%
\pgfpathrectangle{\pgfqpoint{1.069861in}{0.809444in}}{\pgfqpoint{6.135567in}{3.658889in}} %
\pgfusepath{clip}%
\pgfsetrectcap%
\pgfsetroundjoin%
\pgfsetlinewidth{1.505625pt}%
\definecolor{currentstroke}{rgb}{0.121569,0.466667,0.705882}%
\pgfsetstrokecolor{currentstroke}%
\pgfsetdash{}{0pt}%
\pgfpathmoveto{\pgfqpoint{1.348751in}{1.035929in}}%
\pgfpathlineto{\pgfqpoint{1.462583in}{1.081659in}}%
\pgfpathlineto{\pgfqpoint{1.576415in}{1.234494in}}%
\pgfpathlineto{\pgfqpoint{1.690248in}{1.375294in}}%
\pgfpathlineto{\pgfqpoint{1.804080in}{1.401368in}}%
\pgfpathlineto{\pgfqpoint{1.917913in}{1.311112in}}%
\pgfpathlineto{\pgfqpoint{2.031745in}{1.505264in}}%
\pgfpathlineto{\pgfqpoint{2.145577in}{1.565034in}}%
\pgfpathlineto{\pgfqpoint{2.259410in}{1.620391in}}%
\pgfpathlineto{\pgfqpoint{2.373242in}{1.839013in}}%
\pgfpathlineto{\pgfqpoint{2.487075in}{1.928067in}}%
\pgfpathlineto{\pgfqpoint{2.600907in}{2.089727in}}%
\pgfpathlineto{\pgfqpoint{2.714740in}{2.124626in}}%
\pgfpathlineto{\pgfqpoint{2.828572in}{2.134253in}}%
\pgfpathlineto{\pgfqpoint{2.942404in}{2.131044in}}%
\pgfpathlineto{\pgfqpoint{3.056237in}{1.976605in}}%
\pgfpathlineto{\pgfqpoint{3.170069in}{1.884743in}}%
\pgfpathlineto{\pgfqpoint{3.283902in}{1.821363in}}%
\pgfpathlineto{\pgfqpoint{3.397734in}{1.835403in}}%
\pgfpathlineto{\pgfqpoint{3.511566in}{1.836205in}}%
\pgfpathlineto{\pgfqpoint{3.625399in}{1.789673in}}%
\pgfpathlineto{\pgfqpoint{3.739231in}{1.721479in}}%
\pgfpathlineto{\pgfqpoint{3.853064in}{1.647669in}}%
\pgfpathlineto{\pgfqpoint{3.966896in}{1.584289in}}%
\pgfpathlineto{\pgfqpoint{4.080729in}{1.511682in}}%
\pgfpathlineto{\pgfqpoint{4.194561in}{1.413804in}}%
\pgfpathlineto{\pgfqpoint{4.308393in}{1.322344in}}%
\pgfpathlineto{\pgfqpoint{4.422226in}{1.250940in}}%
\pgfpathlineto{\pgfqpoint{4.536058in}{1.181944in}}%
\pgfpathlineto{\pgfqpoint{4.649891in}{1.169910in}}%
\pgfpathlineto{\pgfqpoint{4.763723in}{1.142231in}}%
\pgfpathlineto{\pgfqpoint{4.877556in}{1.114151in}}%
\pgfpathlineto{\pgfqpoint{4.991388in}{1.094495in}}%
\pgfpathlineto{\pgfqpoint{5.105220in}{1.084868in}}%
\pgfpathlineto{\pgfqpoint{5.219053in}{1.076043in}}%
\pgfpathlineto{\pgfqpoint{5.332885in}{1.044353in}}%
\pgfpathlineto{\pgfqpoint{5.446718in}{1.013465in}}%
\pgfpathlineto{\pgfqpoint{5.560550in}{0.976961in}}%
\pgfpathlineto{\pgfqpoint{5.674382in}{0.975758in}}%
\pgfpathlineto{\pgfqpoint{5.788215in}{1.022290in}}%
\pgfpathlineto{\pgfqpoint{5.902047in}{1.128191in}}%
\pgfpathlineto{\pgfqpoint{6.015880in}{1.244121in}}%
\pgfpathlineto{\pgfqpoint{6.129712in}{1.314722in}}%
\pgfpathlineto{\pgfqpoint{6.243545in}{1.269794in}}%
\pgfpathlineto{\pgfqpoint{6.357377in}{1.213233in}}%
\pgfpathlineto{\pgfqpoint{6.471209in}{1.157876in}}%
\pgfpathlineto{\pgfqpoint{6.585042in}{1.116959in}}%
\pgfpathlineto{\pgfqpoint{6.698874in}{1.131801in}}%
\pgfpathlineto{\pgfqpoint{6.812707in}{1.187961in}}%
\pgfpathlineto{\pgfqpoint{6.926539in}{1.248533in}}%
\pgfusepath{stroke}%
\end{pgfscope}%
\begin{pgfscope}%
\pgfpathrectangle{\pgfqpoint{1.069861in}{0.809444in}}{\pgfqpoint{6.135567in}{3.658889in}} %
\pgfusepath{clip}%
\pgfsetrectcap%
\pgfsetroundjoin%
\pgfsetlinewidth{1.505625pt}%
\definecolor{currentstroke}{rgb}{1.000000,0.498039,0.054902}%
\pgfsetstrokecolor{currentstroke}%
\pgfsetdash{}{0pt}%
\pgfpathmoveto{\pgfqpoint{1.348751in}{1.830188in}}%
\pgfpathlineto{\pgfqpoint{1.462583in}{2.262217in}}%
\pgfpathlineto{\pgfqpoint{1.576415in}{2.401413in}}%
\pgfpathlineto{\pgfqpoint{1.690248in}{2.318778in}}%
\pgfpathlineto{\pgfqpoint{1.804080in}{2.289896in}}%
\pgfpathlineto{\pgfqpoint{1.917913in}{2.281873in}}%
\pgfpathlineto{\pgfqpoint{2.031745in}{2.257002in}}%
\pgfpathlineto{\pgfqpoint{2.145577in}{2.283077in}}%
\pgfpathlineto{\pgfqpoint{2.259410in}{2.315569in}}%
\pgfpathlineto{\pgfqpoint{2.373242in}{2.384164in}}%
\pgfpathlineto{\pgfqpoint{2.487075in}{2.429894in}}%
\pgfpathlineto{\pgfqpoint{2.600907in}{2.415854in}}%
\pgfpathlineto{\pgfqpoint{2.714740in}{2.459579in}}%
\pgfpathlineto{\pgfqpoint{2.828572in}{2.513332in}}%
\pgfpathlineto{\pgfqpoint{2.942404in}{2.548632in}}%
\pgfpathlineto{\pgfqpoint{3.056237in}{2.569090in}}%
\pgfpathlineto{\pgfqpoint{3.170069in}{2.550237in}}%
\pgfpathlineto{\pgfqpoint{3.283902in}{2.525366in}}%
\pgfpathlineto{\pgfqpoint{3.397734in}{2.514134in}}%
\pgfpathlineto{\pgfqpoint{3.511566in}{2.469206in}}%
\pgfpathlineto{\pgfqpoint{3.625399in}{2.403419in}}%
\pgfpathlineto{\pgfqpoint{3.739231in}{2.347660in}}%
\pgfpathlineto{\pgfqpoint{3.853064in}{2.309953in}}%
\pgfpathlineto{\pgfqpoint{3.966896in}{2.274251in}}%
\pgfpathlineto{\pgfqpoint{4.080729in}{2.229324in}}%
\pgfpathlineto{\pgfqpoint{4.194561in}{2.193622in}}%
\pgfpathlineto{\pgfqpoint{4.308393in}{2.179582in}}%
\pgfpathlineto{\pgfqpoint{4.422226in}{2.140270in}}%
\pgfpathlineto{\pgfqpoint{4.536058in}{2.116202in}}%
\pgfpathlineto{\pgfqpoint{4.649891in}{2.120614in}}%
\pgfpathlineto{\pgfqpoint{4.763723in}{2.128637in}}%
\pgfpathlineto{\pgfqpoint{4.877556in}{2.151903in}}%
\pgfpathlineto{\pgfqpoint{4.991388in}{2.188808in}}%
\pgfpathlineto{\pgfqpoint{5.105220in}{2.226917in}}%
\pgfpathlineto{\pgfqpoint{5.219053in}{2.244567in}}%
\pgfpathlineto{\pgfqpoint{5.332885in}{2.241358in}}%
\pgfpathlineto{\pgfqpoint{5.446718in}{2.258206in}}%
\pgfpathlineto{\pgfqpoint{5.560550in}{2.300326in}}%
\pgfpathlineto{\pgfqpoint{5.674382in}{2.343248in}}%
\pgfpathlineto{\pgfqpoint{5.788215in}{2.392989in}}%
\pgfpathlineto{\pgfqpoint{5.902047in}{2.454364in}}%
\pgfpathlineto{\pgfqpoint{6.015880in}{2.512128in}}%
\pgfpathlineto{\pgfqpoint{6.129712in}{2.546225in}}%
\pgfpathlineto{\pgfqpoint{6.243545in}{2.561870in}}%
\pgfpathlineto{\pgfqpoint{6.357377in}{2.589548in}}%
\pgfpathlineto{\pgfqpoint{6.471209in}{2.612414in}}%
\pgfpathlineto{\pgfqpoint{6.585042in}{2.635279in}}%
\pgfpathlineto{\pgfqpoint{6.698874in}{2.643301in}}%
\pgfpathlineto{\pgfqpoint{6.812707in}{2.642098in}}%
\pgfpathlineto{\pgfqpoint{6.926539in}{2.666568in}}%
\pgfusepath{stroke}%
\end{pgfscope}%
\begin{pgfscope}%
\pgfpathrectangle{\pgfqpoint{1.069861in}{0.809444in}}{\pgfqpoint{6.135567in}{3.658889in}} %
\pgfusepath{clip}%
\pgfsetrectcap%
\pgfsetroundjoin%
\pgfsetlinewidth{1.505625pt}%
\definecolor{currentstroke}{rgb}{0.172549,0.627451,0.172549}%
\pgfsetstrokecolor{currentstroke}%
\pgfsetdash{}{0pt}%
\pgfpathmoveto{\pgfqpoint{1.348751in}{1.380509in}}%
\pgfpathlineto{\pgfqpoint{1.462583in}{2.386571in}}%
\pgfpathlineto{\pgfqpoint{1.576415in}{2.619634in}}%
\pgfpathlineto{\pgfqpoint{1.690248in}{2.711897in}}%
\pgfpathlineto{\pgfqpoint{1.804080in}{2.843070in}}%
\pgfpathlineto{\pgfqpoint{1.917913in}{3.077336in}}%
\pgfpathlineto{\pgfqpoint{2.031745in}{3.271087in}}%
\pgfpathlineto{\pgfqpoint{2.145577in}{3.418707in}}%
\pgfpathlineto{\pgfqpoint{2.259410in}{3.522603in}}%
\pgfpathlineto{\pgfqpoint{2.373242in}{3.642544in}}%
\pgfpathlineto{\pgfqpoint{2.487075in}{3.805407in}}%
\pgfpathlineto{\pgfqpoint{2.600907in}{3.833086in}}%
\pgfpathlineto{\pgfqpoint{2.714740in}{3.764491in}}%
\pgfpathlineto{\pgfqpoint{2.828572in}{3.788559in}}%
\pgfpathlineto{\pgfqpoint{2.942404in}{3.541055in}}%
\pgfpathlineto{\pgfqpoint{3.056237in}{3.382605in}}%
\pgfpathlineto{\pgfqpoint{3.170069in}{3.321631in}}%
\pgfpathlineto{\pgfqpoint{3.283902in}{3.213323in}}%
\pgfpathlineto{\pgfqpoint{3.397734in}{3.143123in}}%
\pgfpathlineto{\pgfqpoint{3.511566in}{3.062494in}}%
\pgfpathlineto{\pgfqpoint{3.625399in}{3.038024in}}%
\pgfpathlineto{\pgfqpoint{3.739231in}{3.075330in}}%
\pgfpathlineto{\pgfqpoint{3.853064in}{3.113840in}}%
\pgfpathlineto{\pgfqpoint{3.966896in}{3.167593in}}%
\pgfpathlineto{\pgfqpoint{4.080729in}{3.214928in}}%
\pgfpathlineto{\pgfqpoint{4.194561in}{3.189656in}}%
\pgfpathlineto{\pgfqpoint{4.308393in}{3.122264in}}%
\pgfpathlineto{\pgfqpoint{4.422226in}{3.081749in}}%
\pgfpathlineto{\pgfqpoint{4.536058in}{3.022781in}}%
\pgfpathlineto{\pgfqpoint{4.649891in}{2.930920in}}%
\pgfpathlineto{\pgfqpoint{4.763723in}{2.876766in}}%
\pgfpathlineto{\pgfqpoint{4.877556in}{2.834245in}}%
\pgfpathlineto{\pgfqpoint{4.991388in}{2.850691in}}%
\pgfpathlineto{\pgfqpoint{5.105220in}{2.967825in}}%
\pgfpathlineto{\pgfqpoint{5.219053in}{3.215730in}}%
\pgfpathlineto{\pgfqpoint{5.332885in}{3.445183in}}%
\pgfpathlineto{\pgfqpoint{5.446718in}{3.589995in}}%
\pgfpathlineto{\pgfqpoint{5.560550in}{3.723575in}}%
\pgfpathlineto{\pgfqpoint{5.674382in}{3.865980in}}%
\pgfpathlineto{\pgfqpoint{5.788215in}{3.996752in}}%
\pgfpathlineto{\pgfqpoint{5.902047in}{4.050103in}}%
\pgfpathlineto{\pgfqpoint{6.015880in}{4.050103in}}%
\pgfpathlineto{\pgfqpoint{6.129712in}{4.050103in}}%
\pgfpathlineto{\pgfqpoint{6.243545in}{4.103856in}}%
\pgfpathlineto{\pgfqpoint{6.357377in}{4.154801in}}%
\pgfpathlineto{\pgfqpoint{6.471209in}{4.230216in}}%
\pgfpathlineto{\pgfqpoint{6.585042in}{4.302020in}}%
\pgfpathlineto{\pgfqpoint{6.698874in}{4.277551in}}%
\pgfpathlineto{\pgfqpoint{6.812707in}{4.087811in}}%
\pgfpathlineto{\pgfqpoint{6.926539in}{3.732801in}}%
\pgfusepath{stroke}%
\end{pgfscope}%
\begin{pgfscope}%
\pgfsetrectcap%
\pgfsetmiterjoin%
\pgfsetlinewidth{0.803000pt}%
\definecolor{currentstroke}{rgb}{0.000000,0.000000,0.000000}%
\pgfsetstrokecolor{currentstroke}%
\pgfsetdash{}{0pt}%
\pgfpathmoveto{\pgfqpoint{1.069861in}{0.809444in}}%
\pgfpathlineto{\pgfqpoint{1.069861in}{4.468333in}}%
\pgfusepath{stroke}%
\end{pgfscope}%
\begin{pgfscope}%
\pgfsetrectcap%
\pgfsetmiterjoin%
\pgfsetlinewidth{0.803000pt}%
\definecolor{currentstroke}{rgb}{0.000000,0.000000,0.000000}%
\pgfsetstrokecolor{currentstroke}%
\pgfsetdash{}{0pt}%
\pgfpathmoveto{\pgfqpoint{7.205428in}{0.809444in}}%
\pgfpathlineto{\pgfqpoint{7.205428in}{4.468333in}}%
\pgfusepath{stroke}%
\end{pgfscope}%
\begin{pgfscope}%
\pgfsetrectcap%
\pgfsetmiterjoin%
\pgfsetlinewidth{0.803000pt}%
\definecolor{currentstroke}{rgb}{0.000000,0.000000,0.000000}%
\pgfsetstrokecolor{currentstroke}%
\pgfsetdash{}{0pt}%
\pgfpathmoveto{\pgfqpoint{1.069861in}{0.809444in}}%
\pgfpathlineto{\pgfqpoint{7.205428in}{0.809444in}}%
\pgfusepath{stroke}%
\end{pgfscope}%
\begin{pgfscope}%
\pgfsetrectcap%
\pgfsetmiterjoin%
\pgfsetlinewidth{0.803000pt}%
\definecolor{currentstroke}{rgb}{0.000000,0.000000,0.000000}%
\pgfsetstrokecolor{currentstroke}%
\pgfsetdash{}{0pt}%
\pgfpathmoveto{\pgfqpoint{1.069861in}{4.468333in}}%
\pgfpathlineto{\pgfqpoint{7.205428in}{4.468333in}}%
\pgfusepath{stroke}%
\end{pgfscope}%
\begin{pgfscope}%
\pgftext[x=4.137645in,y=4.551667in,,base]{\sffamily\fontsize{19.200000}{23.040000}\selectfont Accelerometer Data (Backpack)}%
\end{pgfscope}%
\begin{pgfscope}%
\pgfsetbuttcap%
\pgfsetmiterjoin%
\definecolor{currentfill}{rgb}{1.000000,1.000000,1.000000}%
\pgfsetfillcolor{currentfill}%
\pgfsetfillopacity{0.800000}%
\pgfsetlinewidth{1.003750pt}%
\definecolor{currentstroke}{rgb}{0.800000,0.800000,0.800000}%
\pgfsetstrokecolor{currentstroke}%
\pgfsetstrokeopacity{0.800000}%
\pgfsetdash{}{0pt}%
\pgfpathmoveto{\pgfqpoint{3.712536in}{3.314645in}}%
\pgfpathlineto{\pgfqpoint{4.562753in}{3.314645in}}%
\pgfpathquadraticcurveto{\pgfqpoint{4.607198in}{3.314645in}}{\pgfqpoint{4.607198in}{3.359089in}}%
\pgfpathlineto{\pgfqpoint{4.607198in}{4.312778in}}%
\pgfpathquadraticcurveto{\pgfqpoint{4.607198in}{4.357222in}}{\pgfqpoint{4.562753in}{4.357222in}}%
\pgfpathlineto{\pgfqpoint{3.712536in}{4.357222in}}%
\pgfpathquadraticcurveto{\pgfqpoint{3.668092in}{4.357222in}}{\pgfqpoint{3.668092in}{4.312778in}}%
\pgfpathlineto{\pgfqpoint{3.668092in}{3.359089in}}%
\pgfpathquadraticcurveto{\pgfqpoint{3.668092in}{3.314645in}}{\pgfqpoint{3.712536in}{3.314645in}}%
\pgfpathclose%
\pgfusepath{stroke,fill}%
\end{pgfscope}%
\begin{pgfscope}%
\pgfsetrectcap%
\pgfsetroundjoin%
\pgfsetlinewidth{1.505625pt}%
\definecolor{currentstroke}{rgb}{0.121569,0.466667,0.705882}%
\pgfsetstrokecolor{currentstroke}%
\pgfsetdash{}{0pt}%
\pgfpathmoveto{\pgfqpoint{3.756981in}{4.174779in}}%
\pgfpathlineto{\pgfqpoint{4.201425in}{4.174779in}}%
\pgfusepath{stroke}%
\end{pgfscope}%
\begin{pgfscope}%
\pgftext[x=4.379203in,y=4.097001in,left,base]{\sffamily\fontsize{16.000000}{19.200000}\selectfont X}%
\end{pgfscope}%
\begin{pgfscope}%
\pgfsetrectcap%
\pgfsetroundjoin%
\pgfsetlinewidth{1.505625pt}%
\definecolor{currentstroke}{rgb}{1.000000,0.498039,0.054902}%
\pgfsetstrokecolor{currentstroke}%
\pgfsetdash{}{0pt}%
\pgfpathmoveto{\pgfqpoint{3.756981in}{3.849475in}}%
\pgfpathlineto{\pgfqpoint{4.201425in}{3.849475in}}%
\pgfusepath{stroke}%
\end{pgfscope}%
\begin{pgfscope}%
\pgftext[x=4.379203in,y=3.771697in,left,base]{\sffamily\fontsize{16.000000}{19.200000}\selectfont Y}%
\end{pgfscope}%
\begin{pgfscope}%
\pgfsetrectcap%
\pgfsetroundjoin%
\pgfsetlinewidth{1.505625pt}%
\definecolor{currentstroke}{rgb}{0.172549,0.627451,0.172549}%
\pgfsetstrokecolor{currentstroke}%
\pgfsetdash{}{0pt}%
\pgfpathmoveto{\pgfqpoint{3.756981in}{3.524171in}}%
\pgfpathlineto{\pgfqpoint{4.201425in}{3.524171in}}%
\pgfusepath{stroke}%
\end{pgfscope}%
\begin{pgfscope}%
\pgftext[x=4.379203in,y=3.446394in,left,base]{\sffamily\fontsize{16.000000}{19.200000}\selectfont Z}%
\end{pgfscope}%
\end{pgfpicture}%
\makeatother%
\endgroup%
}
  \scalebox{0.1}{%% Creator: Matplotlib, PGF backend
%%
%% To include the figure in your LaTeX document, write
%%   \input{<filename>.pgf}
%%
%% Make sure the required packages are loaded in your preamble
%%   \usepackage{pgf}
%%
%% Figures using additional raster images can only be included by \input if
%% they are in the same directory as the main LaTeX file. For loading figures
%% from other directories you can use the `import` package
%%   \usepackage{import}
%% and then include the figures with
%%   \import{<path to file>}{<filename>.pgf}
%%
%% Matplotlib used the following preamble
%%   \usepackage{fontspec}
%%   \setmainfont{Times New Roman}
%%   \setsansfont{Lucida Grande}
%%   \setmonofont{Andale Mono}
%%
\begingroup%
\makeatletter%
\begin{pgfpicture}%
\pgfpathrectangle{\pgfpointorigin}{\pgfqpoint{7.500000in}{5.000000in}}%
\pgfusepath{use as bounding box, clip}%
\begin{pgfscope}%
\pgfsetbuttcap%
\pgfsetmiterjoin%
\definecolor{currentfill}{rgb}{1.000000,1.000000,1.000000}%
\pgfsetfillcolor{currentfill}%
\pgfsetlinewidth{0.000000pt}%
\definecolor{currentstroke}{rgb}{1.000000,1.000000,1.000000}%
\pgfsetstrokecolor{currentstroke}%
\pgfsetdash{}{0pt}%
\pgfpathmoveto{\pgfqpoint{0.000000in}{0.000000in}}%
\pgfpathlineto{\pgfqpoint{7.500000in}{0.000000in}}%
\pgfpathlineto{\pgfqpoint{7.500000in}{5.000000in}}%
\pgfpathlineto{\pgfqpoint{0.000000in}{5.000000in}}%
\pgfpathclose%
\pgfusepath{fill}%
\end{pgfscope}%
\begin{pgfscope}%
\pgfsetbuttcap%
\pgfsetmiterjoin%
\definecolor{currentfill}{rgb}{1.000000,1.000000,1.000000}%
\pgfsetfillcolor{currentfill}%
\pgfsetlinewidth{0.000000pt}%
\definecolor{currentstroke}{rgb}{0.000000,0.000000,0.000000}%
\pgfsetstrokecolor{currentstroke}%
\pgfsetstrokeopacity{0.000000}%
\pgfsetdash{}{0pt}%
\pgfpathmoveto{\pgfqpoint{1.069861in}{0.809444in}}%
\pgfpathlineto{\pgfqpoint{7.205428in}{0.809444in}}%
\pgfpathlineto{\pgfqpoint{7.205428in}{4.468333in}}%
\pgfpathlineto{\pgfqpoint{1.069861in}{4.468333in}}%
\pgfpathclose%
\pgfusepath{fill}%
\end{pgfscope}%
\begin{pgfscope}%
\pgfsetbuttcap%
\pgfsetroundjoin%
\definecolor{currentfill}{rgb}{0.000000,0.000000,0.000000}%
\pgfsetfillcolor{currentfill}%
\pgfsetlinewidth{0.803000pt}%
\definecolor{currentstroke}{rgb}{0.000000,0.000000,0.000000}%
\pgfsetstrokecolor{currentstroke}%
\pgfsetdash{}{0pt}%
\pgfsys@defobject{currentmarker}{\pgfqpoint{0.000000in}{-0.048611in}}{\pgfqpoint{0.000000in}{0.000000in}}{%
\pgfpathmoveto{\pgfqpoint{0.000000in}{0.000000in}}%
\pgfpathlineto{\pgfqpoint{0.000000in}{-0.048611in}}%
\pgfusepath{stroke,fill}%
}%
\begin{pgfscope}%
\pgfsys@transformshift{1.348751in}{0.809444in}%
\pgfsys@useobject{currentmarker}{}%
\end{pgfscope}%
\end{pgfscope}%
\begin{pgfscope}%
\pgftext[x=1.348751in,y=0.712222in,,top]{\sffamily\fontsize{16.000000}{19.200000}\selectfont 0}%
\end{pgfscope}%
\begin{pgfscope}%
\pgfsetbuttcap%
\pgfsetroundjoin%
\definecolor{currentfill}{rgb}{0.000000,0.000000,0.000000}%
\pgfsetfillcolor{currentfill}%
\pgfsetlinewidth{0.803000pt}%
\definecolor{currentstroke}{rgb}{0.000000,0.000000,0.000000}%
\pgfsetstrokecolor{currentstroke}%
\pgfsetdash{}{0pt}%
\pgfsys@defobject{currentmarker}{\pgfqpoint{0.000000in}{-0.048611in}}{\pgfqpoint{0.000000in}{0.000000in}}{%
\pgfpathmoveto{\pgfqpoint{0.000000in}{0.000000in}}%
\pgfpathlineto{\pgfqpoint{0.000000in}{-0.048611in}}%
\pgfusepath{stroke,fill}%
}%
\begin{pgfscope}%
\pgfsys@transformshift{2.487075in}{0.809444in}%
\pgfsys@useobject{currentmarker}{}%
\end{pgfscope}%
\end{pgfscope}%
\begin{pgfscope}%
\pgftext[x=2.487075in,y=0.712222in,,top]{\sffamily\fontsize{16.000000}{19.200000}\selectfont 100}%
\end{pgfscope}%
\begin{pgfscope}%
\pgfsetbuttcap%
\pgfsetroundjoin%
\definecolor{currentfill}{rgb}{0.000000,0.000000,0.000000}%
\pgfsetfillcolor{currentfill}%
\pgfsetlinewidth{0.803000pt}%
\definecolor{currentstroke}{rgb}{0.000000,0.000000,0.000000}%
\pgfsetstrokecolor{currentstroke}%
\pgfsetdash{}{0pt}%
\pgfsys@defobject{currentmarker}{\pgfqpoint{0.000000in}{-0.048611in}}{\pgfqpoint{0.000000in}{0.000000in}}{%
\pgfpathmoveto{\pgfqpoint{0.000000in}{0.000000in}}%
\pgfpathlineto{\pgfqpoint{0.000000in}{-0.048611in}}%
\pgfusepath{stroke,fill}%
}%
\begin{pgfscope}%
\pgfsys@transformshift{3.625399in}{0.809444in}%
\pgfsys@useobject{currentmarker}{}%
\end{pgfscope}%
\end{pgfscope}%
\begin{pgfscope}%
\pgftext[x=3.625399in,y=0.712222in,,top]{\sffamily\fontsize{16.000000}{19.200000}\selectfont 200}%
\end{pgfscope}%
\begin{pgfscope}%
\pgfsetbuttcap%
\pgfsetroundjoin%
\definecolor{currentfill}{rgb}{0.000000,0.000000,0.000000}%
\pgfsetfillcolor{currentfill}%
\pgfsetlinewidth{0.803000pt}%
\definecolor{currentstroke}{rgb}{0.000000,0.000000,0.000000}%
\pgfsetstrokecolor{currentstroke}%
\pgfsetdash{}{0pt}%
\pgfsys@defobject{currentmarker}{\pgfqpoint{0.000000in}{-0.048611in}}{\pgfqpoint{0.000000in}{0.000000in}}{%
\pgfpathmoveto{\pgfqpoint{0.000000in}{0.000000in}}%
\pgfpathlineto{\pgfqpoint{0.000000in}{-0.048611in}}%
\pgfusepath{stroke,fill}%
}%
\begin{pgfscope}%
\pgfsys@transformshift{4.763723in}{0.809444in}%
\pgfsys@useobject{currentmarker}{}%
\end{pgfscope}%
\end{pgfscope}%
\begin{pgfscope}%
\pgftext[x=4.763723in,y=0.712222in,,top]{\sffamily\fontsize{16.000000}{19.200000}\selectfont 300}%
\end{pgfscope}%
\begin{pgfscope}%
\pgfsetbuttcap%
\pgfsetroundjoin%
\definecolor{currentfill}{rgb}{0.000000,0.000000,0.000000}%
\pgfsetfillcolor{currentfill}%
\pgfsetlinewidth{0.803000pt}%
\definecolor{currentstroke}{rgb}{0.000000,0.000000,0.000000}%
\pgfsetstrokecolor{currentstroke}%
\pgfsetdash{}{0pt}%
\pgfsys@defobject{currentmarker}{\pgfqpoint{0.000000in}{-0.048611in}}{\pgfqpoint{0.000000in}{0.000000in}}{%
\pgfpathmoveto{\pgfqpoint{0.000000in}{0.000000in}}%
\pgfpathlineto{\pgfqpoint{0.000000in}{-0.048611in}}%
\pgfusepath{stroke,fill}%
}%
\begin{pgfscope}%
\pgfsys@transformshift{5.902047in}{0.809444in}%
\pgfsys@useobject{currentmarker}{}%
\end{pgfscope}%
\end{pgfscope}%
\begin{pgfscope}%
\pgftext[x=5.902047in,y=0.712222in,,top]{\sffamily\fontsize{16.000000}{19.200000}\selectfont 400}%
\end{pgfscope}%
\begin{pgfscope}%
\pgfsetbuttcap%
\pgfsetroundjoin%
\definecolor{currentfill}{rgb}{0.000000,0.000000,0.000000}%
\pgfsetfillcolor{currentfill}%
\pgfsetlinewidth{0.803000pt}%
\definecolor{currentstroke}{rgb}{0.000000,0.000000,0.000000}%
\pgfsetstrokecolor{currentstroke}%
\pgfsetdash{}{0pt}%
\pgfsys@defobject{currentmarker}{\pgfqpoint{0.000000in}{-0.048611in}}{\pgfqpoint{0.000000in}{0.000000in}}{%
\pgfpathmoveto{\pgfqpoint{0.000000in}{0.000000in}}%
\pgfpathlineto{\pgfqpoint{0.000000in}{-0.048611in}}%
\pgfusepath{stroke,fill}%
}%
\begin{pgfscope}%
\pgfsys@transformshift{7.040371in}{0.809444in}%
\pgfsys@useobject{currentmarker}{}%
\end{pgfscope}%
\end{pgfscope}%
\begin{pgfscope}%
\pgftext[x=7.040371in,y=0.712222in,,top]{\sffamily\fontsize{16.000000}{19.200000}\selectfont 500}%
\end{pgfscope}%
\begin{pgfscope}%
\pgftext[x=4.137645in,y=0.442474in,,top]{\sffamily\fontsize{16.000000}{19.200000}\selectfont Time (ms)}%
\end{pgfscope}%
\begin{pgfscope}%
\pgfsetbuttcap%
\pgfsetroundjoin%
\definecolor{currentfill}{rgb}{0.000000,0.000000,0.000000}%
\pgfsetfillcolor{currentfill}%
\pgfsetlinewidth{0.803000pt}%
\definecolor{currentstroke}{rgb}{0.000000,0.000000,0.000000}%
\pgfsetstrokecolor{currentstroke}%
\pgfsetdash{}{0pt}%
\pgfsys@defobject{currentmarker}{\pgfqpoint{-0.048611in}{0.000000in}}{\pgfqpoint{0.000000in}{0.000000in}}{%
\pgfpathmoveto{\pgfqpoint{0.000000in}{0.000000in}}%
\pgfpathlineto{\pgfqpoint{-0.048611in}{0.000000in}}%
\pgfusepath{stroke,fill}%
}%
\begin{pgfscope}%
\pgfsys@transformshift{1.069861in}{0.955144in}%
\pgfsys@useobject{currentmarker}{}%
\end{pgfscope}%
\end{pgfscope}%
\begin{pgfscope}%
\pgftext[x=0.514848in,y=0.869478in,left,base]{\sffamily\fontsize{16.000000}{19.200000}\selectfont −20}%
\end{pgfscope}%
\begin{pgfscope}%
\pgfsetbuttcap%
\pgfsetroundjoin%
\definecolor{currentfill}{rgb}{0.000000,0.000000,0.000000}%
\pgfsetfillcolor{currentfill}%
\pgfsetlinewidth{0.803000pt}%
\definecolor{currentstroke}{rgb}{0.000000,0.000000,0.000000}%
\pgfsetstrokecolor{currentstroke}%
\pgfsetdash{}{0pt}%
\pgfsys@defobject{currentmarker}{\pgfqpoint{-0.048611in}{0.000000in}}{\pgfqpoint{0.000000in}{0.000000in}}{%
\pgfpathmoveto{\pgfqpoint{0.000000in}{0.000000in}}%
\pgfpathlineto{\pgfqpoint{-0.048611in}{0.000000in}}%
\pgfusepath{stroke,fill}%
}%
\begin{pgfscope}%
\pgfsys@transformshift{1.069861in}{1.509256in}%
\pgfsys@useobject{currentmarker}{}%
\end{pgfscope}%
\end{pgfscope}%
\begin{pgfscope}%
\pgftext[x=0.514848in,y=1.423589in,left,base]{\sffamily\fontsize{16.000000}{19.200000}\selectfont −15}%
\end{pgfscope}%
\begin{pgfscope}%
\pgfsetbuttcap%
\pgfsetroundjoin%
\definecolor{currentfill}{rgb}{0.000000,0.000000,0.000000}%
\pgfsetfillcolor{currentfill}%
\pgfsetlinewidth{0.803000pt}%
\definecolor{currentstroke}{rgb}{0.000000,0.000000,0.000000}%
\pgfsetstrokecolor{currentstroke}%
\pgfsetdash{}{0pt}%
\pgfsys@defobject{currentmarker}{\pgfqpoint{-0.048611in}{0.000000in}}{\pgfqpoint{0.000000in}{0.000000in}}{%
\pgfpathmoveto{\pgfqpoint{0.000000in}{0.000000in}}%
\pgfpathlineto{\pgfqpoint{-0.048611in}{0.000000in}}%
\pgfusepath{stroke,fill}%
}%
\begin{pgfscope}%
\pgfsys@transformshift{1.069861in}{2.063367in}%
\pgfsys@useobject{currentmarker}{}%
\end{pgfscope}%
\end{pgfscope}%
\begin{pgfscope}%
\pgftext[x=0.514848in,y=1.977701in,left,base]{\sffamily\fontsize{16.000000}{19.200000}\selectfont −10}%
\end{pgfscope}%
\begin{pgfscope}%
\pgfsetbuttcap%
\pgfsetroundjoin%
\definecolor{currentfill}{rgb}{0.000000,0.000000,0.000000}%
\pgfsetfillcolor{currentfill}%
\pgfsetlinewidth{0.803000pt}%
\definecolor{currentstroke}{rgb}{0.000000,0.000000,0.000000}%
\pgfsetstrokecolor{currentstroke}%
\pgfsetdash{}{0pt}%
\pgfsys@defobject{currentmarker}{\pgfqpoint{-0.048611in}{0.000000in}}{\pgfqpoint{0.000000in}{0.000000in}}{%
\pgfpathmoveto{\pgfqpoint{0.000000in}{0.000000in}}%
\pgfpathlineto{\pgfqpoint{-0.048611in}{0.000000in}}%
\pgfusepath{stroke,fill}%
}%
\begin{pgfscope}%
\pgfsys@transformshift{1.069861in}{2.617478in}%
\pgfsys@useobject{currentmarker}{}%
\end{pgfscope}%
\end{pgfscope}%
\begin{pgfscope}%
\pgftext[x=0.655365in,y=2.531812in,left,base]{\sffamily\fontsize{16.000000}{19.200000}\selectfont −5}%
\end{pgfscope}%
\begin{pgfscope}%
\pgfsetbuttcap%
\pgfsetroundjoin%
\definecolor{currentfill}{rgb}{0.000000,0.000000,0.000000}%
\pgfsetfillcolor{currentfill}%
\pgfsetlinewidth{0.803000pt}%
\definecolor{currentstroke}{rgb}{0.000000,0.000000,0.000000}%
\pgfsetstrokecolor{currentstroke}%
\pgfsetdash{}{0pt}%
\pgfsys@defobject{currentmarker}{\pgfqpoint{-0.048611in}{0.000000in}}{\pgfqpoint{0.000000in}{0.000000in}}{%
\pgfpathmoveto{\pgfqpoint{0.000000in}{0.000000in}}%
\pgfpathlineto{\pgfqpoint{-0.048611in}{0.000000in}}%
\pgfusepath{stroke,fill}%
}%
\begin{pgfscope}%
\pgfsys@transformshift{1.069861in}{3.171590in}%
\pgfsys@useobject{currentmarker}{}%
\end{pgfscope}%
\end{pgfscope}%
\begin{pgfscope}%
\pgftext[x=0.832122in,y=3.085924in,left,base]{\sffamily\fontsize{16.000000}{19.200000}\selectfont 0}%
\end{pgfscope}%
\begin{pgfscope}%
\pgfsetbuttcap%
\pgfsetroundjoin%
\definecolor{currentfill}{rgb}{0.000000,0.000000,0.000000}%
\pgfsetfillcolor{currentfill}%
\pgfsetlinewidth{0.803000pt}%
\definecolor{currentstroke}{rgb}{0.000000,0.000000,0.000000}%
\pgfsetstrokecolor{currentstroke}%
\pgfsetdash{}{0pt}%
\pgfsys@defobject{currentmarker}{\pgfqpoint{-0.048611in}{0.000000in}}{\pgfqpoint{0.000000in}{0.000000in}}{%
\pgfpathmoveto{\pgfqpoint{0.000000in}{0.000000in}}%
\pgfpathlineto{\pgfqpoint{-0.048611in}{0.000000in}}%
\pgfusepath{stroke,fill}%
}%
\begin{pgfscope}%
\pgfsys@transformshift{1.069861in}{3.725701in}%
\pgfsys@useobject{currentmarker}{}%
\end{pgfscope}%
\end{pgfscope}%
\begin{pgfscope}%
\pgftext[x=0.832122in,y=3.640035in,left,base]{\sffamily\fontsize{16.000000}{19.200000}\selectfont 5}%
\end{pgfscope}%
\begin{pgfscope}%
\pgfsetbuttcap%
\pgfsetroundjoin%
\definecolor{currentfill}{rgb}{0.000000,0.000000,0.000000}%
\pgfsetfillcolor{currentfill}%
\pgfsetlinewidth{0.803000pt}%
\definecolor{currentstroke}{rgb}{0.000000,0.000000,0.000000}%
\pgfsetstrokecolor{currentstroke}%
\pgfsetdash{}{0pt}%
\pgfsys@defobject{currentmarker}{\pgfqpoint{-0.048611in}{0.000000in}}{\pgfqpoint{0.000000in}{0.000000in}}{%
\pgfpathmoveto{\pgfqpoint{0.000000in}{0.000000in}}%
\pgfpathlineto{\pgfqpoint{-0.048611in}{0.000000in}}%
\pgfusepath{stroke,fill}%
}%
\begin{pgfscope}%
\pgfsys@transformshift{1.069861in}{4.279813in}%
\pgfsys@useobject{currentmarker}{}%
\end{pgfscope}%
\end{pgfscope}%
\begin{pgfscope}%
\pgftext[x=0.691606in,y=4.194146in,left,base]{\sffamily\fontsize{16.000000}{19.200000}\selectfont 10}%
\end{pgfscope}%
\begin{pgfscope}%
\pgftext[x=0.459292in,y=2.638889in,,bottom,rotate=90.000000]{\sffamily\fontsize{16.000000}{19.200000}\selectfont Acceleration (m/s\^2)}%
\end{pgfscope}%
\begin{pgfscope}%
\pgfpathrectangle{\pgfqpoint{1.069861in}{0.809444in}}{\pgfqpoint{6.135567in}{3.658889in}} %
\pgfusepath{clip}%
\pgfsetrectcap%
\pgfsetroundjoin%
\pgfsetlinewidth{1.505625pt}%
\definecolor{currentstroke}{rgb}{0.121569,0.466667,0.705882}%
\pgfsetstrokecolor{currentstroke}%
\pgfsetdash{}{0pt}%
\pgfpathmoveto{\pgfqpoint{1.348751in}{1.343897in}}%
\pgfpathlineto{\pgfqpoint{1.462583in}{1.363804in}}%
\pgfpathlineto{\pgfqpoint{1.576415in}{1.538717in}}%
\pgfpathlineto{\pgfqpoint{1.690248in}{1.585165in}}%
\pgfpathlineto{\pgfqpoint{1.804080in}{1.594986in}}%
\pgfpathlineto{\pgfqpoint{1.917913in}{1.697970in}}%
\pgfpathlineto{\pgfqpoint{2.031745in}{1.692396in}}%
\pgfpathlineto{\pgfqpoint{2.145577in}{1.997896in}}%
\pgfpathlineto{\pgfqpoint{2.259410in}{2.227751in}}%
\pgfpathlineto{\pgfqpoint{2.373242in}{2.545726in}}%
\pgfpathlineto{\pgfqpoint{2.487075in}{2.732052in}}%
\pgfpathlineto{\pgfqpoint{2.600907in}{3.151418in}}%
\pgfpathlineto{\pgfqpoint{2.714740in}{3.259710in}}%
\pgfpathlineto{\pgfqpoint{2.828572in}{3.189373in}}%
\pgfpathlineto{\pgfqpoint{2.942404in}{3.025873in}}%
\pgfpathlineto{\pgfqpoint{3.056237in}{3.009417in}}%
\pgfpathlineto{\pgfqpoint{3.170069in}{3.009417in}}%
\pgfpathlineto{\pgfqpoint{3.283902in}{3.007559in}}%
\pgfpathlineto{\pgfqpoint{3.397734in}{2.995350in}}%
\pgfpathlineto{\pgfqpoint{3.511566in}{2.916254in}}%
\pgfpathlineto{\pgfqpoint{3.625399in}{2.740014in}}%
\pgfpathlineto{\pgfqpoint{3.739231in}{2.512017in}}%
\pgfpathlineto{\pgfqpoint{3.853064in}{2.293310in}}%
\pgfpathlineto{\pgfqpoint{3.966896in}{2.087078in}}%
\pgfpathlineto{\pgfqpoint{4.080729in}{1.883499in}}%
\pgfpathlineto{\pgfqpoint{4.194561in}{1.802811in}}%
\pgfpathlineto{\pgfqpoint{4.308393in}{1.786620in}}%
\pgfpathlineto{\pgfqpoint{4.422226in}{1.728228in}}%
\pgfpathlineto{\pgfqpoint{4.536058in}{1.538717in}}%
\pgfpathlineto{\pgfqpoint{4.649891in}{1.482713in}}%
\pgfpathlineto{\pgfqpoint{4.763723in}{1.469972in}}%
\pgfpathlineto{\pgfqpoint{4.877556in}{1.471830in}}%
\pgfpathlineto{\pgfqpoint{4.991388in}{1.583838in}}%
\pgfpathlineto{\pgfqpoint{5.105220in}{1.704870in}}%
\pgfpathlineto{\pgfqpoint{5.219053in}{1.663730in}}%
\pgfpathlineto{\pgfqpoint{5.332885in}{1.476343in}}%
\pgfpathlineto{\pgfqpoint{5.446718in}{1.539778in}}%
\pgfpathlineto{\pgfqpoint{5.560550in}{1.885888in}}%
\pgfpathlineto{\pgfqpoint{5.674382in}{1.725573in}}%
\pgfpathlineto{\pgfqpoint{5.788215in}{1.173762in}}%
\pgfpathlineto{\pgfqpoint{5.902047in}{1.432548in}}%
\pgfpathlineto{\pgfqpoint{6.015880in}{2.025234in}}%
\pgfpathlineto{\pgfqpoint{6.129712in}{2.091059in}}%
\pgfpathlineto{\pgfqpoint{6.243545in}{2.014883in}}%
\pgfpathlineto{\pgfqpoint{6.357377in}{2.058943in}}%
\pgfpathlineto{\pgfqpoint{6.471209in}{2.131403in}}%
\pgfpathlineto{\pgfqpoint{6.585042in}{2.113620in}}%
\pgfpathlineto{\pgfqpoint{6.698874in}{2.045672in}}%
\pgfpathlineto{\pgfqpoint{6.812707in}{1.932337in}}%
\pgfpathlineto{\pgfqpoint{6.926539in}{1.961268in}}%
\pgfusepath{stroke}%
\end{pgfscope}%
\begin{pgfscope}%
\pgfpathrectangle{\pgfqpoint{1.069861in}{0.809444in}}{\pgfqpoint{6.135567in}{3.658889in}} %
\pgfusepath{clip}%
\pgfsetrectcap%
\pgfsetroundjoin%
\pgfsetlinewidth{1.505625pt}%
\definecolor{currentstroke}{rgb}{1.000000,0.498039,0.054902}%
\pgfsetstrokecolor{currentstroke}%
\pgfsetdash{}{0pt}%
\pgfpathmoveto{\pgfqpoint{1.348751in}{3.262364in}}%
\pgfpathlineto{\pgfqpoint{1.462583in}{3.064625in}}%
\pgfpathlineto{\pgfqpoint{1.576415in}{2.753551in}}%
\pgfpathlineto{\pgfqpoint{1.690248in}{2.461853in}}%
\pgfpathlineto{\pgfqpoint{1.804080in}{2.203332in}}%
\pgfpathlineto{\pgfqpoint{1.917913in}{1.909776in}}%
\pgfpathlineto{\pgfqpoint{2.031745in}{1.734598in}}%
\pgfpathlineto{\pgfqpoint{2.145577in}{1.782904in}}%
\pgfpathlineto{\pgfqpoint{2.259410in}{1.795379in}}%
\pgfpathlineto{\pgfqpoint{2.373242in}{1.915084in}}%
\pgfpathlineto{\pgfqpoint{2.487075in}{2.185814in}}%
\pgfpathlineto{\pgfqpoint{2.600907in}{2.377980in}}%
\pgfpathlineto{\pgfqpoint{2.714740in}{2.581027in}}%
\pgfpathlineto{\pgfqpoint{2.828572in}{2.778501in}}%
\pgfpathlineto{\pgfqpoint{2.942404in}{2.963234in}}%
\pgfpathlineto{\pgfqpoint{3.056237in}{3.047903in}}%
\pgfpathlineto{\pgfqpoint{3.170069in}{3.114259in}}%
\pgfpathlineto{\pgfqpoint{3.283902in}{3.165751in}}%
\pgfpathlineto{\pgfqpoint{3.397734in}{3.183003in}}%
\pgfpathlineto{\pgfqpoint{3.511566in}{3.172121in}}%
\pgfpathlineto{\pgfqpoint{3.625399in}{3.172386in}}%
\pgfpathlineto{\pgfqpoint{3.739231in}{3.203706in}}%
\pgfpathlineto{\pgfqpoint{3.853064in}{3.247500in}}%
\pgfpathlineto{\pgfqpoint{3.966896in}{3.295276in}}%
\pgfpathlineto{\pgfqpoint{4.080729in}{3.364286in}}%
\pgfpathlineto{\pgfqpoint{4.194561in}{3.377822in}}%
\pgfpathlineto{\pgfqpoint{4.308393in}{3.364020in}}%
\pgfpathlineto{\pgfqpoint{4.422226in}{3.396136in}}%
\pgfpathlineto{\pgfqpoint{4.536058in}{3.411796in}}%
\pgfpathlineto{\pgfqpoint{4.649891in}{3.493546in}}%
\pgfpathlineto{\pgfqpoint{4.763723in}{3.660496in}}%
\pgfpathlineto{\pgfqpoint{4.877556in}{3.723401in}}%
\pgfpathlineto{\pgfqpoint{4.991388in}{3.666070in}}%
\pgfpathlineto{\pgfqpoint{5.105220in}{3.660762in}}%
\pgfpathlineto{\pgfqpoint{5.219053in}{3.782325in}}%
\pgfpathlineto{\pgfqpoint{5.332885in}{4.042703in}}%
\pgfpathlineto{\pgfqpoint{5.446718in}{4.109324in}}%
\pgfpathlineto{\pgfqpoint{5.560550in}{3.916097in}}%
\pgfpathlineto{\pgfqpoint{5.674382in}{3.711192in}}%
\pgfpathlineto{\pgfqpoint{5.788215in}{3.761356in}}%
\pgfpathlineto{\pgfqpoint{5.902047in}{3.611659in}}%
\pgfpathlineto{\pgfqpoint{6.015880in}{3.386847in}}%
\pgfpathlineto{\pgfqpoint{6.129712in}{3.341194in}}%
\pgfpathlineto{\pgfqpoint{6.243545in}{3.403037in}}%
\pgfpathlineto{\pgfqpoint{6.357377in}{3.437542in}}%
\pgfpathlineto{\pgfqpoint{6.471209in}{3.420024in}}%
\pgfpathlineto{\pgfqpoint{6.585042in}{3.445505in}}%
\pgfpathlineto{\pgfqpoint{6.698874in}{3.434092in}}%
\pgfpathlineto{\pgfqpoint{6.812707in}{3.310936in}}%
\pgfpathlineto{\pgfqpoint{6.926539in}{2.970400in}}%
\pgfusepath{stroke}%
\end{pgfscope}%
\begin{pgfscope}%
\pgfpathrectangle{\pgfqpoint{1.069861in}{0.809444in}}{\pgfqpoint{6.135567in}{3.658889in}} %
\pgfusepath{clip}%
\pgfsetrectcap%
\pgfsetroundjoin%
\pgfsetlinewidth{1.505625pt}%
\definecolor{currentstroke}{rgb}{0.172549,0.627451,0.172549}%
\pgfsetstrokecolor{currentstroke}%
\pgfsetdash{}{0pt}%
\pgfpathmoveto{\pgfqpoint{1.348751in}{3.148233in}}%
\pgfpathlineto{\pgfqpoint{1.462583in}{3.069668in}}%
\pgfpathlineto{\pgfqpoint{1.576415in}{3.131246in}}%
\pgfpathlineto{\pgfqpoint{1.690248in}{3.423475in}}%
\pgfpathlineto{\pgfqpoint{1.804080in}{3.726852in}}%
\pgfpathlineto{\pgfqpoint{1.917913in}{3.844168in}}%
\pgfpathlineto{\pgfqpoint{2.031745in}{4.109324in}}%
\pgfpathlineto{\pgfqpoint{2.145577in}{4.302020in}}%
\pgfpathlineto{\pgfqpoint{2.259410in}{4.199302in}}%
\pgfpathlineto{\pgfqpoint{2.373242in}{3.887432in}}%
\pgfpathlineto{\pgfqpoint{2.487075in}{3.700575in}}%
\pgfpathlineto{\pgfqpoint{2.600907in}{3.502305in}}%
\pgfpathlineto{\pgfqpoint{2.714740in}{3.431438in}}%
\pgfpathlineto{\pgfqpoint{2.828572in}{3.399852in}}%
\pgfpathlineto{\pgfqpoint{2.942404in}{3.309874in}}%
\pgfpathlineto{\pgfqpoint{3.056237in}{3.256259in}}%
\pgfpathlineto{\pgfqpoint{3.170069in}{3.238742in}}%
\pgfpathlineto{\pgfqpoint{3.283902in}{3.248297in}}%
\pgfpathlineto{\pgfqpoint{3.397734in}{3.231310in}}%
\pgfpathlineto{\pgfqpoint{3.511566in}{3.190169in}}%
\pgfpathlineto{\pgfqpoint{3.625399in}{3.220693in}}%
\pgfpathlineto{\pgfqpoint{3.739231in}{3.291295in}}%
\pgfpathlineto{\pgfqpoint{3.853064in}{3.343052in}}%
\pgfpathlineto{\pgfqpoint{3.966896in}{3.350484in}}%
\pgfpathlineto{\pgfqpoint{4.080729in}{3.094618in}}%
\pgfpathlineto{\pgfqpoint{4.194561in}{2.815394in}}%
\pgfpathlineto{\pgfqpoint{4.308393in}{2.834239in}}%
\pgfpathlineto{\pgfqpoint{4.422226in}{3.150091in}}%
\pgfpathlineto{\pgfqpoint{4.536058in}{3.386847in}}%
\pgfpathlineto{\pgfqpoint{4.649891in}{3.413123in}}%
\pgfpathlineto{\pgfqpoint{4.763723in}{3.391359in}}%
\pgfpathlineto{\pgfqpoint{4.877556in}{3.294745in}}%
\pgfpathlineto{\pgfqpoint{4.991388in}{3.304301in}}%
\pgfpathlineto{\pgfqpoint{5.105220in}{3.271388in}}%
\pgfpathlineto{\pgfqpoint{5.219053in}{2.822561in}}%
\pgfpathlineto{\pgfqpoint{5.332885in}{1.581715in}}%
\pgfpathlineto{\pgfqpoint{5.446718in}{0.975758in}}%
\pgfpathlineto{\pgfqpoint{5.560550in}{1.682575in}}%
\pgfpathlineto{\pgfqpoint{5.674382in}{2.881219in}}%
\pgfpathlineto{\pgfqpoint{5.788215in}{3.124079in}}%
\pgfpathlineto{\pgfqpoint{5.902047in}{3.356854in}}%
\pgfpathlineto{\pgfqpoint{6.015880in}{3.470720in}}%
\pgfpathlineto{\pgfqpoint{6.129712in}{3.319430in}}%
\pgfpathlineto{\pgfqpoint{6.243545in}{3.294745in}}%
\pgfpathlineto{\pgfqpoint{6.357377in}{3.351811in}}%
\pgfpathlineto{\pgfqpoint{6.471209in}{3.468331in}}%
\pgfpathlineto{\pgfqpoint{6.585042in}{3.485849in}}%
\pgfpathlineto{\pgfqpoint{6.698874in}{3.561494in}}%
\pgfpathlineto{\pgfqpoint{6.812707in}{3.783917in}}%
\pgfpathlineto{\pgfqpoint{6.926539in}{3.891413in}}%
\pgfusepath{stroke}%
\end{pgfscope}%
\begin{pgfscope}%
\pgfsetrectcap%
\pgfsetmiterjoin%
\pgfsetlinewidth{0.803000pt}%
\definecolor{currentstroke}{rgb}{0.000000,0.000000,0.000000}%
\pgfsetstrokecolor{currentstroke}%
\pgfsetdash{}{0pt}%
\pgfpathmoveto{\pgfqpoint{1.069861in}{0.809444in}}%
\pgfpathlineto{\pgfqpoint{1.069861in}{4.468333in}}%
\pgfusepath{stroke}%
\end{pgfscope}%
\begin{pgfscope}%
\pgfsetrectcap%
\pgfsetmiterjoin%
\pgfsetlinewidth{0.803000pt}%
\definecolor{currentstroke}{rgb}{0.000000,0.000000,0.000000}%
\pgfsetstrokecolor{currentstroke}%
\pgfsetdash{}{0pt}%
\pgfpathmoveto{\pgfqpoint{7.205428in}{0.809444in}}%
\pgfpathlineto{\pgfqpoint{7.205428in}{4.468333in}}%
\pgfusepath{stroke}%
\end{pgfscope}%
\begin{pgfscope}%
\pgfsetrectcap%
\pgfsetmiterjoin%
\pgfsetlinewidth{0.803000pt}%
\definecolor{currentstroke}{rgb}{0.000000,0.000000,0.000000}%
\pgfsetstrokecolor{currentstroke}%
\pgfsetdash{}{0pt}%
\pgfpathmoveto{\pgfqpoint{1.069861in}{0.809444in}}%
\pgfpathlineto{\pgfqpoint{7.205428in}{0.809444in}}%
\pgfusepath{stroke}%
\end{pgfscope}%
\begin{pgfscope}%
\pgfsetrectcap%
\pgfsetmiterjoin%
\pgfsetlinewidth{0.803000pt}%
\definecolor{currentstroke}{rgb}{0.000000,0.000000,0.000000}%
\pgfsetstrokecolor{currentstroke}%
\pgfsetdash{}{0pt}%
\pgfpathmoveto{\pgfqpoint{1.069861in}{4.468333in}}%
\pgfpathlineto{\pgfqpoint{7.205428in}{4.468333in}}%
\pgfusepath{stroke}%
\end{pgfscope}%
\begin{pgfscope}%
\pgftext[x=4.137645in,y=4.551667in,,base]{\sffamily\fontsize{19.200000}{23.040000}\selectfont Accelerometer Data (Pocket)}%
\end{pgfscope}%
\begin{pgfscope}%
\pgfsetbuttcap%
\pgfsetmiterjoin%
\definecolor{currentfill}{rgb}{1.000000,1.000000,1.000000}%
\pgfsetfillcolor{currentfill}%
\pgfsetfillopacity{0.800000}%
\pgfsetlinewidth{1.003750pt}%
\definecolor{currentstroke}{rgb}{0.800000,0.800000,0.800000}%
\pgfsetstrokecolor{currentstroke}%
\pgfsetstrokeopacity{0.800000}%
\pgfsetdash{}{0pt}%
\pgfpathmoveto{\pgfqpoint{6.199656in}{0.920556in}}%
\pgfpathlineto{\pgfqpoint{7.049873in}{0.920556in}}%
\pgfpathquadraticcurveto{\pgfqpoint{7.094317in}{0.920556in}}{\pgfqpoint{7.094317in}{0.965000in}}%
\pgfpathlineto{\pgfqpoint{7.094317in}{1.918689in}}%
\pgfpathquadraticcurveto{\pgfqpoint{7.094317in}{1.963133in}}{\pgfqpoint{7.049873in}{1.963133in}}%
\pgfpathlineto{\pgfqpoint{6.199656in}{1.963133in}}%
\pgfpathquadraticcurveto{\pgfqpoint{6.155211in}{1.963133in}}{\pgfqpoint{6.155211in}{1.918689in}}%
\pgfpathlineto{\pgfqpoint{6.155211in}{0.965000in}}%
\pgfpathquadraticcurveto{\pgfqpoint{6.155211in}{0.920556in}}{\pgfqpoint{6.199656in}{0.920556in}}%
\pgfpathclose%
\pgfusepath{stroke,fill}%
\end{pgfscope}%
\begin{pgfscope}%
\pgfsetrectcap%
\pgfsetroundjoin%
\pgfsetlinewidth{1.505625pt}%
\definecolor{currentstroke}{rgb}{0.121569,0.466667,0.705882}%
\pgfsetstrokecolor{currentstroke}%
\pgfsetdash{}{0pt}%
\pgfpathmoveto{\pgfqpoint{6.244100in}{1.780689in}}%
\pgfpathlineto{\pgfqpoint{6.688545in}{1.780689in}}%
\pgfusepath{stroke}%
\end{pgfscope}%
\begin{pgfscope}%
\pgftext[x=6.866322in,y=1.702912in,left,base]{\sffamily\fontsize{16.000000}{19.200000}\selectfont X}%
\end{pgfscope}%
\begin{pgfscope}%
\pgfsetrectcap%
\pgfsetroundjoin%
\pgfsetlinewidth{1.505625pt}%
\definecolor{currentstroke}{rgb}{1.000000,0.498039,0.054902}%
\pgfsetstrokecolor{currentstroke}%
\pgfsetdash{}{0pt}%
\pgfpathmoveto{\pgfqpoint{6.244100in}{1.455386in}}%
\pgfpathlineto{\pgfqpoint{6.688545in}{1.455386in}}%
\pgfusepath{stroke}%
\end{pgfscope}%
\begin{pgfscope}%
\pgftext[x=6.866322in,y=1.377608in,left,base]{\sffamily\fontsize{16.000000}{19.200000}\selectfont Y}%
\end{pgfscope}%
\begin{pgfscope}%
\pgfsetrectcap%
\pgfsetroundjoin%
\pgfsetlinewidth{1.505625pt}%
\definecolor{currentstroke}{rgb}{0.172549,0.627451,0.172549}%
\pgfsetstrokecolor{currentstroke}%
\pgfsetdash{}{0pt}%
\pgfpathmoveto{\pgfqpoint{6.244100in}{1.130082in}}%
\pgfpathlineto{\pgfqpoint{6.688545in}{1.130082in}}%
\pgfusepath{stroke}%
\end{pgfscope}%
\begin{pgfscope}%
\pgftext[x=6.866322in,y=1.052304in,left,base]{\sffamily\fontsize{16.000000}{19.200000}\selectfont Z}%
\end{pgfscope}%
\end{pgfpicture}%
\makeatother%
\endgroup%
}
  \scalebox{0.1}{%% Creator: Matplotlib, PGF backend
%%
%% To include the figure in your LaTeX document, write
%%   \input{<filename>.pgf}
%%
%% Make sure the required packages are loaded in your preamble
%%   \usepackage{pgf}
%%
%% Figures using additional raster images can only be included by \input if
%% they are in the same directory as the main LaTeX file. For loading figures
%% from other directories you can use the `import` package
%%   \usepackage{import}
%% and then include the figures with
%%   \import{<path to file>}{<filename>.pgf}
%%
%% Matplotlib used the following preamble
%%   \usepackage{fontspec}
%%   \setmainfont{Times New Roman}
%%   \setsansfont{Lucida Grande}
%%   \setmonofont{Andale Mono}
%%
\begingroup%
\makeatletter%
\begin{pgfpicture}%
\pgfpathrectangle{\pgfpointorigin}{\pgfqpoint{7.500000in}{5.000000in}}%
\pgfusepath{use as bounding box, clip}%
\begin{pgfscope}%
\pgfsetbuttcap%
\pgfsetmiterjoin%
\definecolor{currentfill}{rgb}{1.000000,1.000000,1.000000}%
\pgfsetfillcolor{currentfill}%
\pgfsetlinewidth{0.000000pt}%
\definecolor{currentstroke}{rgb}{1.000000,1.000000,1.000000}%
\pgfsetstrokecolor{currentstroke}%
\pgfsetdash{}{0pt}%
\pgfpathmoveto{\pgfqpoint{0.000000in}{0.000000in}}%
\pgfpathlineto{\pgfqpoint{7.500000in}{0.000000in}}%
\pgfpathlineto{\pgfqpoint{7.500000in}{5.000000in}}%
\pgfpathlineto{\pgfqpoint{0.000000in}{5.000000in}}%
\pgfpathclose%
\pgfusepath{fill}%
\end{pgfscope}%
\begin{pgfscope}%
\pgfsetbuttcap%
\pgfsetmiterjoin%
\definecolor{currentfill}{rgb}{1.000000,1.000000,1.000000}%
\pgfsetfillcolor{currentfill}%
\pgfsetlinewidth{0.000000pt}%
\definecolor{currentstroke}{rgb}{0.000000,0.000000,0.000000}%
\pgfsetstrokecolor{currentstroke}%
\pgfsetstrokeopacity{0.000000}%
\pgfsetdash{}{0pt}%
\pgfpathmoveto{\pgfqpoint{0.617014in}{0.580556in}}%
\pgfpathlineto{\pgfqpoint{7.301389in}{0.580556in}}%
\pgfpathlineto{\pgfqpoint{7.301389in}{4.627778in}}%
\pgfpathlineto{\pgfqpoint{0.617014in}{4.627778in}}%
\pgfpathclose%
\pgfusepath{fill}%
\end{pgfscope}%
\begin{pgfscope}%
\pgfsetbuttcap%
\pgfsetroundjoin%
\definecolor{currentfill}{rgb}{0.000000,0.000000,0.000000}%
\pgfsetfillcolor{currentfill}%
\pgfsetlinewidth{0.803000pt}%
\definecolor{currentstroke}{rgb}{0.000000,0.000000,0.000000}%
\pgfsetstrokecolor{currentstroke}%
\pgfsetdash{}{0pt}%
\pgfsys@defobject{currentmarker}{\pgfqpoint{0.000000in}{-0.048611in}}{\pgfqpoint{0.000000in}{0.000000in}}{%
\pgfpathmoveto{\pgfqpoint{0.000000in}{0.000000in}}%
\pgfpathlineto{\pgfqpoint{0.000000in}{-0.048611in}}%
\pgfusepath{stroke,fill}%
}%
\begin{pgfscope}%
\pgfsys@transformshift{0.920849in}{0.580556in}%
\pgfsys@useobject{currentmarker}{}%
\end{pgfscope}%
\end{pgfscope}%
\begin{pgfscope}%
\pgftext[x=0.920849in,y=0.483333in,,top]{\sffamily\fontsize{10.000000}{12.000000}\selectfont 0}%
\end{pgfscope}%
\begin{pgfscope}%
\pgfsetbuttcap%
\pgfsetroundjoin%
\definecolor{currentfill}{rgb}{0.000000,0.000000,0.000000}%
\pgfsetfillcolor{currentfill}%
\pgfsetlinewidth{0.803000pt}%
\definecolor{currentstroke}{rgb}{0.000000,0.000000,0.000000}%
\pgfsetstrokecolor{currentstroke}%
\pgfsetdash{}{0pt}%
\pgfsys@defobject{currentmarker}{\pgfqpoint{0.000000in}{-0.048611in}}{\pgfqpoint{0.000000in}{0.000000in}}{%
\pgfpathmoveto{\pgfqpoint{0.000000in}{0.000000in}}%
\pgfpathlineto{\pgfqpoint{0.000000in}{-0.048611in}}%
\pgfusepath{stroke,fill}%
}%
\begin{pgfscope}%
\pgfsys@transformshift{2.160993in}{0.580556in}%
\pgfsys@useobject{currentmarker}{}%
\end{pgfscope}%
\end{pgfscope}%
\begin{pgfscope}%
\pgftext[x=2.160993in,y=0.483333in,,top]{\sffamily\fontsize{10.000000}{12.000000}\selectfont 100}%
\end{pgfscope}%
\begin{pgfscope}%
\pgfsetbuttcap%
\pgfsetroundjoin%
\definecolor{currentfill}{rgb}{0.000000,0.000000,0.000000}%
\pgfsetfillcolor{currentfill}%
\pgfsetlinewidth{0.803000pt}%
\definecolor{currentstroke}{rgb}{0.000000,0.000000,0.000000}%
\pgfsetstrokecolor{currentstroke}%
\pgfsetdash{}{0pt}%
\pgfsys@defobject{currentmarker}{\pgfqpoint{0.000000in}{-0.048611in}}{\pgfqpoint{0.000000in}{0.000000in}}{%
\pgfpathmoveto{\pgfqpoint{0.000000in}{0.000000in}}%
\pgfpathlineto{\pgfqpoint{0.000000in}{-0.048611in}}%
\pgfusepath{stroke,fill}%
}%
\begin{pgfscope}%
\pgfsys@transformshift{3.401137in}{0.580556in}%
\pgfsys@useobject{currentmarker}{}%
\end{pgfscope}%
\end{pgfscope}%
\begin{pgfscope}%
\pgftext[x=3.401137in,y=0.483333in,,top]{\sffamily\fontsize{10.000000}{12.000000}\selectfont 200}%
\end{pgfscope}%
\begin{pgfscope}%
\pgfsetbuttcap%
\pgfsetroundjoin%
\definecolor{currentfill}{rgb}{0.000000,0.000000,0.000000}%
\pgfsetfillcolor{currentfill}%
\pgfsetlinewidth{0.803000pt}%
\definecolor{currentstroke}{rgb}{0.000000,0.000000,0.000000}%
\pgfsetstrokecolor{currentstroke}%
\pgfsetdash{}{0pt}%
\pgfsys@defobject{currentmarker}{\pgfqpoint{0.000000in}{-0.048611in}}{\pgfqpoint{0.000000in}{0.000000in}}{%
\pgfpathmoveto{\pgfqpoint{0.000000in}{0.000000in}}%
\pgfpathlineto{\pgfqpoint{0.000000in}{-0.048611in}}%
\pgfusepath{stroke,fill}%
}%
\begin{pgfscope}%
\pgfsys@transformshift{4.641280in}{0.580556in}%
\pgfsys@useobject{currentmarker}{}%
\end{pgfscope}%
\end{pgfscope}%
\begin{pgfscope}%
\pgftext[x=4.641280in,y=0.483333in,,top]{\sffamily\fontsize{10.000000}{12.000000}\selectfont 300}%
\end{pgfscope}%
\begin{pgfscope}%
\pgfsetbuttcap%
\pgfsetroundjoin%
\definecolor{currentfill}{rgb}{0.000000,0.000000,0.000000}%
\pgfsetfillcolor{currentfill}%
\pgfsetlinewidth{0.803000pt}%
\definecolor{currentstroke}{rgb}{0.000000,0.000000,0.000000}%
\pgfsetstrokecolor{currentstroke}%
\pgfsetdash{}{0pt}%
\pgfsys@defobject{currentmarker}{\pgfqpoint{0.000000in}{-0.048611in}}{\pgfqpoint{0.000000in}{0.000000in}}{%
\pgfpathmoveto{\pgfqpoint{0.000000in}{0.000000in}}%
\pgfpathlineto{\pgfqpoint{0.000000in}{-0.048611in}}%
\pgfusepath{stroke,fill}%
}%
\begin{pgfscope}%
\pgfsys@transformshift{5.881424in}{0.580556in}%
\pgfsys@useobject{currentmarker}{}%
\end{pgfscope}%
\end{pgfscope}%
\begin{pgfscope}%
\pgftext[x=5.881424in,y=0.483333in,,top]{\sffamily\fontsize{10.000000}{12.000000}\selectfont 400}%
\end{pgfscope}%
\begin{pgfscope}%
\pgfsetbuttcap%
\pgfsetroundjoin%
\definecolor{currentfill}{rgb}{0.000000,0.000000,0.000000}%
\pgfsetfillcolor{currentfill}%
\pgfsetlinewidth{0.803000pt}%
\definecolor{currentstroke}{rgb}{0.000000,0.000000,0.000000}%
\pgfsetstrokecolor{currentstroke}%
\pgfsetdash{}{0pt}%
\pgfsys@defobject{currentmarker}{\pgfqpoint{0.000000in}{-0.048611in}}{\pgfqpoint{0.000000in}{0.000000in}}{%
\pgfpathmoveto{\pgfqpoint{0.000000in}{0.000000in}}%
\pgfpathlineto{\pgfqpoint{0.000000in}{-0.048611in}}%
\pgfusepath{stroke,fill}%
}%
\begin{pgfscope}%
\pgfsys@transformshift{7.121568in}{0.580556in}%
\pgfsys@useobject{currentmarker}{}%
\end{pgfscope}%
\end{pgfscope}%
\begin{pgfscope}%
\pgftext[x=7.121568in,y=0.483333in,,top]{\sffamily\fontsize{10.000000}{12.000000}\selectfont 500}%
\end{pgfscope}%
\begin{pgfscope}%
\pgftext[x=3.959201in,y=0.293907in,,top]{\sffamily\fontsize{10.000000}{12.000000}\selectfont Time (ms)}%
\end{pgfscope}%
\begin{pgfscope}%
\pgfsetbuttcap%
\pgfsetroundjoin%
\definecolor{currentfill}{rgb}{0.000000,0.000000,0.000000}%
\pgfsetfillcolor{currentfill}%
\pgfsetlinewidth{0.803000pt}%
\definecolor{currentstroke}{rgb}{0.000000,0.000000,0.000000}%
\pgfsetstrokecolor{currentstroke}%
\pgfsetdash{}{0pt}%
\pgfsys@defobject{currentmarker}{\pgfqpoint{-0.048611in}{0.000000in}}{\pgfqpoint{0.000000in}{0.000000in}}{%
\pgfpathmoveto{\pgfqpoint{0.000000in}{0.000000in}}%
\pgfpathlineto{\pgfqpoint{-0.048611in}{0.000000in}}%
\pgfusepath{stroke,fill}%
}%
\begin{pgfscope}%
\pgfsys@transformshift{0.617014in}{0.847201in}%
\pgfsys@useobject{currentmarker}{}%
\end{pgfscope}%
\end{pgfscope}%
\begin{pgfscope}%
\pgftext[x=0.431969in,y=0.793660in,left,base]{\sffamily\fontsize{10.000000}{12.000000}\selectfont 0}%
\end{pgfscope}%
\begin{pgfscope}%
\pgfsetbuttcap%
\pgfsetroundjoin%
\definecolor{currentfill}{rgb}{0.000000,0.000000,0.000000}%
\pgfsetfillcolor{currentfill}%
\pgfsetlinewidth{0.803000pt}%
\definecolor{currentstroke}{rgb}{0.000000,0.000000,0.000000}%
\pgfsetstrokecolor{currentstroke}%
\pgfsetdash{}{0pt}%
\pgfsys@defobject{currentmarker}{\pgfqpoint{-0.048611in}{0.000000in}}{\pgfqpoint{0.000000in}{0.000000in}}{%
\pgfpathmoveto{\pgfqpoint{0.000000in}{0.000000in}}%
\pgfpathlineto{\pgfqpoint{-0.048611in}{0.000000in}}%
\pgfusepath{stroke,fill}%
}%
\begin{pgfscope}%
\pgfsys@transformshift{0.617014in}{1.581710in}%
\pgfsys@useobject{currentmarker}{}%
\end{pgfscope}%
\end{pgfscope}%
\begin{pgfscope}%
\pgftext[x=0.431969in,y=1.528169in,left,base]{\sffamily\fontsize{10.000000}{12.000000}\selectfont 2}%
\end{pgfscope}%
\begin{pgfscope}%
\pgfsetbuttcap%
\pgfsetroundjoin%
\definecolor{currentfill}{rgb}{0.000000,0.000000,0.000000}%
\pgfsetfillcolor{currentfill}%
\pgfsetlinewidth{0.803000pt}%
\definecolor{currentstroke}{rgb}{0.000000,0.000000,0.000000}%
\pgfsetstrokecolor{currentstroke}%
\pgfsetdash{}{0pt}%
\pgfsys@defobject{currentmarker}{\pgfqpoint{-0.048611in}{0.000000in}}{\pgfqpoint{0.000000in}{0.000000in}}{%
\pgfpathmoveto{\pgfqpoint{0.000000in}{0.000000in}}%
\pgfpathlineto{\pgfqpoint{-0.048611in}{0.000000in}}%
\pgfusepath{stroke,fill}%
}%
\begin{pgfscope}%
\pgfsys@transformshift{0.617014in}{2.316220in}%
\pgfsys@useobject{currentmarker}{}%
\end{pgfscope}%
\end{pgfscope}%
\begin{pgfscope}%
\pgftext[x=0.431969in,y=2.262679in,left,base]{\sffamily\fontsize{10.000000}{12.000000}\selectfont 4}%
\end{pgfscope}%
\begin{pgfscope}%
\pgfsetbuttcap%
\pgfsetroundjoin%
\definecolor{currentfill}{rgb}{0.000000,0.000000,0.000000}%
\pgfsetfillcolor{currentfill}%
\pgfsetlinewidth{0.803000pt}%
\definecolor{currentstroke}{rgb}{0.000000,0.000000,0.000000}%
\pgfsetstrokecolor{currentstroke}%
\pgfsetdash{}{0pt}%
\pgfsys@defobject{currentmarker}{\pgfqpoint{-0.048611in}{0.000000in}}{\pgfqpoint{0.000000in}{0.000000in}}{%
\pgfpathmoveto{\pgfqpoint{0.000000in}{0.000000in}}%
\pgfpathlineto{\pgfqpoint{-0.048611in}{0.000000in}}%
\pgfusepath{stroke,fill}%
}%
\begin{pgfscope}%
\pgfsys@transformshift{0.617014in}{3.050730in}%
\pgfsys@useobject{currentmarker}{}%
\end{pgfscope}%
\end{pgfscope}%
\begin{pgfscope}%
\pgftext[x=0.431969in,y=2.997188in,left,base]{\sffamily\fontsize{10.000000}{12.000000}\selectfont 6}%
\end{pgfscope}%
\begin{pgfscope}%
\pgfsetbuttcap%
\pgfsetroundjoin%
\definecolor{currentfill}{rgb}{0.000000,0.000000,0.000000}%
\pgfsetfillcolor{currentfill}%
\pgfsetlinewidth{0.803000pt}%
\definecolor{currentstroke}{rgb}{0.000000,0.000000,0.000000}%
\pgfsetstrokecolor{currentstroke}%
\pgfsetdash{}{0pt}%
\pgfsys@defobject{currentmarker}{\pgfqpoint{-0.048611in}{0.000000in}}{\pgfqpoint{0.000000in}{0.000000in}}{%
\pgfpathmoveto{\pgfqpoint{0.000000in}{0.000000in}}%
\pgfpathlineto{\pgfqpoint{-0.048611in}{0.000000in}}%
\pgfusepath{stroke,fill}%
}%
\begin{pgfscope}%
\pgfsys@transformshift{0.617014in}{3.785239in}%
\pgfsys@useobject{currentmarker}{}%
\end{pgfscope}%
\end{pgfscope}%
\begin{pgfscope}%
\pgftext[x=0.431969in,y=3.731698in,left,base]{\sffamily\fontsize{10.000000}{12.000000}\selectfont 8}%
\end{pgfscope}%
\begin{pgfscope}%
\pgfsetbuttcap%
\pgfsetroundjoin%
\definecolor{currentfill}{rgb}{0.000000,0.000000,0.000000}%
\pgfsetfillcolor{currentfill}%
\pgfsetlinewidth{0.803000pt}%
\definecolor{currentstroke}{rgb}{0.000000,0.000000,0.000000}%
\pgfsetstrokecolor{currentstroke}%
\pgfsetdash{}{0pt}%
\pgfsys@defobject{currentmarker}{\pgfqpoint{-0.048611in}{0.000000in}}{\pgfqpoint{0.000000in}{0.000000in}}{%
\pgfpathmoveto{\pgfqpoint{0.000000in}{0.000000in}}%
\pgfpathlineto{\pgfqpoint{-0.048611in}{0.000000in}}%
\pgfusepath{stroke,fill}%
}%
\begin{pgfscope}%
\pgfsys@transformshift{0.617014in}{4.519749in}%
\pgfsys@useobject{currentmarker}{}%
\end{pgfscope}%
\end{pgfscope}%
\begin{pgfscope}%
\pgftext[x=0.344146in,y=4.466207in,left,base]{\sffamily\fontsize{10.000000}{12.000000}\selectfont 10}%
\end{pgfscope}%
\begin{pgfscope}%
\pgftext[x=0.288591in,y=2.604167in,,bottom,rotate=90.000000]{\sffamily\fontsize{10.000000}{12.000000}\selectfont Acceleration (g)}%
\end{pgfscope}%
\begin{pgfscope}%
\pgfpathrectangle{\pgfqpoint{0.617014in}{0.580556in}}{\pgfqpoint{6.684375in}{4.047222in}} %
\pgfusepath{clip}%
\pgfsetrectcap%
\pgfsetroundjoin%
\pgfsetlinewidth{1.505625pt}%
\definecolor{currentstroke}{rgb}{0.121569,0.466667,0.705882}%
\pgfsetstrokecolor{currentstroke}%
\pgfsetdash{}{0pt}%
\pgfpathmoveto{\pgfqpoint{0.920849in}{0.918447in}}%
\pgfpathlineto{\pgfqpoint{1.044863in}{0.961547in}}%
\pgfpathlineto{\pgfqpoint{1.168878in}{0.980018in}}%
\pgfpathlineto{\pgfqpoint{1.292892in}{0.969463in}}%
\pgfpathlineto{\pgfqpoint{1.416907in}{0.942196in}}%
\pgfpathlineto{\pgfqpoint{1.540921in}{0.921086in}}%
\pgfpathlineto{\pgfqpoint{1.664935in}{0.889421in}}%
\pgfpathlineto{\pgfqpoint{1.788950in}{0.854238in}}%
\pgfpathlineto{\pgfqpoint{1.912964in}{0.848081in}}%
\pgfpathlineto{\pgfqpoint{2.036979in}{0.854238in}}%
\pgfpathlineto{\pgfqpoint{2.160993in}{0.846321in}}%
\pgfpathlineto{\pgfqpoint{2.285007in}{0.829609in}}%
\pgfpathlineto{\pgfqpoint{2.409022in}{0.828730in}}%
\pgfpathlineto{\pgfqpoint{2.533036in}{0.821693in}}%
\pgfpathlineto{\pgfqpoint{2.657050in}{0.800583in}}%
\pgfpathlineto{\pgfqpoint{2.781065in}{0.790028in}}%
\pgfpathlineto{\pgfqpoint{2.905079in}{0.816416in}}%
\pgfpathlineto{\pgfqpoint{3.029094in}{0.824332in}}%
\pgfpathlineto{\pgfqpoint{3.153108in}{0.790028in}}%
\pgfpathlineto{\pgfqpoint{3.277122in}{0.764520in}}%
\pgfpathlineto{\pgfqpoint{3.401137in}{0.792667in}}%
\pgfpathlineto{\pgfqpoint{3.525151in}{0.825211in}}%
\pgfpathlineto{\pgfqpoint{3.649165in}{0.842803in}}%
\pgfpathlineto{\pgfqpoint{3.773180in}{0.877107in}}%
\pgfpathlineto{\pgfqpoint{3.897194in}{0.921965in}}%
\pgfpathlineto{\pgfqpoint{4.021209in}{0.962426in}}%
\pgfpathlineto{\pgfqpoint{4.145223in}{0.961547in}}%
\pgfpathlineto{\pgfqpoint{4.269237in}{0.935159in}}%
\pgfpathlineto{\pgfqpoint{4.393252in}{0.910531in}}%
\pgfpathlineto{\pgfqpoint{4.517266in}{0.910531in}}%
\pgfpathlineto{\pgfqpoint{4.641280in}{0.924604in}}%
\pgfpathlineto{\pgfqpoint{4.765295in}{0.919327in}}%
\pgfpathlineto{\pgfqpoint{4.889309in}{0.912290in}}%
\pgfpathlineto{\pgfqpoint{5.013324in}{0.892939in}}%
\pgfpathlineto{\pgfqpoint{5.137338in}{0.920206in}}%
\pgfpathlineto{\pgfqpoint{5.261352in}{0.972981in}}%
\pgfpathlineto{\pgfqpoint{5.385367in}{0.978259in}}%
\pgfpathlineto{\pgfqpoint{5.509381in}{0.951871in}}%
\pgfpathlineto{\pgfqpoint{5.633395in}{0.937798in}}%
\pgfpathlineto{\pgfqpoint{5.757410in}{0.936918in}}%
\pgfpathlineto{\pgfqpoint{5.881424in}{0.909651in}}%
\pgfpathlineto{\pgfqpoint{6.005439in}{0.854238in}}%
\pgfpathlineto{\pgfqpoint{6.129453in}{0.836646in}}%
\pgfpathlineto{\pgfqpoint{6.253467in}{0.847201in}}%
\pgfpathlineto{\pgfqpoint{6.377482in}{0.848960in}}%
\pgfpathlineto{\pgfqpoint{6.501496in}{0.876227in}}%
\pgfpathlineto{\pgfqpoint{6.625511in}{0.890300in}}%
\pgfpathlineto{\pgfqpoint{6.749525in}{0.892939in}}%
\pgfpathlineto{\pgfqpoint{6.873539in}{0.913170in}}%
\pgfpathlineto{\pgfqpoint{6.997554in}{0.942196in}}%
\pgfusepath{stroke}%
\end{pgfscope}%
\begin{pgfscope}%
\pgfpathrectangle{\pgfqpoint{0.617014in}{0.580556in}}{\pgfqpoint{6.684375in}{4.047222in}} %
\pgfusepath{clip}%
\pgfsetrectcap%
\pgfsetroundjoin%
\pgfsetlinewidth{1.505625pt}%
\definecolor{currentstroke}{rgb}{1.000000,0.498039,0.054902}%
\pgfsetstrokecolor{currentstroke}%
\pgfsetdash{}{0pt}%
\pgfpathmoveto{\pgfqpoint{0.920849in}{2.480585in}}%
\pgfpathlineto{\pgfqpoint{1.044863in}{2.419894in}}%
\pgfpathlineto{\pgfqpoint{1.168878in}{2.365360in}}%
\pgfpathlineto{\pgfqpoint{1.292892in}{2.339852in}}%
\pgfpathlineto{\pgfqpoint{1.416907in}{2.380313in}}%
\pgfpathlineto{\pgfqpoint{1.540921in}{2.438365in}}%
\pgfpathlineto{\pgfqpoint{1.664935in}{2.482345in}}%
\pgfpathlineto{\pgfqpoint{1.788950in}{2.514889in}}%
\pgfpathlineto{\pgfqpoint{1.912964in}{2.536879in}}%
\pgfpathlineto{\pgfqpoint{2.036979in}{2.535999in}}%
\pgfpathlineto{\pgfqpoint{2.160993in}{2.536879in}}%
\pgfpathlineto{\pgfqpoint{2.285007in}{2.550072in}}%
\pgfpathlineto{\pgfqpoint{2.409022in}{2.581737in}}%
\pgfpathlineto{\pgfqpoint{2.533036in}{2.611643in}}%
\pgfpathlineto{\pgfqpoint{2.657050in}{2.621319in}}%
\pgfpathlineto{\pgfqpoint{2.781065in}{2.604607in}}%
\pgfpathlineto{\pgfqpoint{2.905079in}{2.594931in}}%
\pgfpathlineto{\pgfqpoint{3.029094in}{2.614282in}}%
\pgfpathlineto{\pgfqpoint{3.153108in}{2.660020in}}%
\pgfpathlineto{\pgfqpoint{3.277122in}{2.683769in}}%
\pgfpathlineto{\pgfqpoint{3.401137in}{2.696963in}}%
\pgfpathlineto{\pgfqpoint{3.525151in}{2.674973in}}%
\pgfpathlineto{\pgfqpoint{3.649165in}{2.628355in}}%
\pgfpathlineto{\pgfqpoint{3.773180in}{2.594931in}}%
\pgfpathlineto{\pgfqpoint{3.897194in}{2.575580in}}%
\pgfpathlineto{\pgfqpoint{4.021209in}{2.539517in}}%
\pgfpathlineto{\pgfqpoint{4.145223in}{2.517528in}}%
\pgfpathlineto{\pgfqpoint{4.269237in}{2.511371in}}%
\pgfpathlineto{\pgfqpoint{4.393252in}{2.513130in}}%
\pgfpathlineto{\pgfqpoint{4.517266in}{2.515769in}}%
\pgfpathlineto{\pgfqpoint{4.641280in}{2.507853in}}%
\pgfpathlineto{\pgfqpoint{4.765295in}{2.500816in}}%
\pgfpathlineto{\pgfqpoint{4.889309in}{2.494659in}}%
\pgfpathlineto{\pgfqpoint{5.013324in}{2.493779in}}%
\pgfpathlineto{\pgfqpoint{5.137338in}{2.483224in}}%
\pgfpathlineto{\pgfqpoint{5.261352in}{2.465632in}}%
\pgfpathlineto{\pgfqpoint{5.385367in}{2.460355in}}%
\pgfpathlineto{\pgfqpoint{5.509381in}{2.462114in}}%
\pgfpathlineto{\pgfqpoint{5.633395in}{2.459475in}}%
\pgfpathlineto{\pgfqpoint{5.757410in}{2.460355in}}%
\pgfpathlineto{\pgfqpoint{5.881424in}{2.470030in}}%
\pgfpathlineto{\pgfqpoint{6.005439in}{2.491140in}}%
\pgfpathlineto{\pgfqpoint{6.129453in}{2.530722in}}%
\pgfpathlineto{\pgfqpoint{6.253467in}{2.542156in}}%
\pgfpathlineto{\pgfqpoint{6.377482in}{2.538638in}}%
\pgfpathlineto{\pgfqpoint{6.501496in}{2.542156in}}%
\pgfpathlineto{\pgfqpoint{6.625511in}{2.533360in}}%
\pgfpathlineto{\pgfqpoint{6.749525in}{2.521046in}}%
\pgfpathlineto{\pgfqpoint{6.873539in}{2.495538in}}%
\pgfpathlineto{\pgfqpoint{6.997554in}{2.475308in}}%
\pgfusepath{stroke}%
\end{pgfscope}%
\begin{pgfscope}%
\pgfpathrectangle{\pgfqpoint{0.617014in}{0.580556in}}{\pgfqpoint{6.684375in}{4.047222in}} %
\pgfusepath{clip}%
\pgfsetrectcap%
\pgfsetroundjoin%
\pgfsetlinewidth{1.505625pt}%
\definecolor{currentstroke}{rgb}{0.172549,0.627451,0.172549}%
\pgfsetstrokecolor{currentstroke}%
\pgfsetdash{}{0pt}%
\pgfpathmoveto{\pgfqpoint{0.920849in}{4.143876in}}%
\pgfpathlineto{\pgfqpoint{1.044863in}{4.286368in}}%
\pgfpathlineto{\pgfqpoint{1.168878in}{4.408630in}}%
\pgfpathlineto{\pgfqpoint{1.292892in}{4.443813in}}%
\pgfpathlineto{\pgfqpoint{1.416907in}{4.392798in}}%
\pgfpathlineto{\pgfqpoint{1.540921in}{4.324190in}}%
\pgfpathlineto{\pgfqpoint{1.664935in}{4.277572in}}%
\pgfpathlineto{\pgfqpoint{1.788950in}{4.238871in}}%
\pgfpathlineto{\pgfqpoint{1.912964in}{4.252944in}}%
\pgfpathlineto{\pgfqpoint{2.036979in}{4.275813in}}%
\pgfpathlineto{\pgfqpoint{2.160993in}{4.258221in}}%
\pgfpathlineto{\pgfqpoint{2.285007in}{4.196651in}}%
\pgfpathlineto{\pgfqpoint{2.409022in}{4.087583in}}%
\pgfpathlineto{\pgfqpoint{2.533036in}{4.032169in}}%
\pgfpathlineto{\pgfqpoint{2.657050in}{4.053279in}}%
\pgfpathlineto{\pgfqpoint{2.781065in}{4.155310in}}%
\pgfpathlineto{\pgfqpoint{2.905079in}{4.276693in}}%
\pgfpathlineto{\pgfqpoint{3.029094in}{4.314515in}}%
\pgfpathlineto{\pgfqpoint{3.153108in}{4.235352in}}%
\pgfpathlineto{\pgfqpoint{3.277122in}{4.073509in}}%
\pgfpathlineto{\pgfqpoint{3.401137in}{3.991708in}}%
\pgfpathlineto{\pgfqpoint{3.525151in}{3.980273in}}%
\pgfpathlineto{\pgfqpoint{3.649165in}{4.013697in}}%
\pgfpathlineto{\pgfqpoint{3.773180in}{4.033048in}}%
\pgfpathlineto{\pgfqpoint{3.897194in}{4.016336in}}%
\pgfpathlineto{\pgfqpoint{4.021209in}{4.069991in}}%
\pgfpathlineto{\pgfqpoint{4.145223in}{4.136839in}}%
\pgfpathlineto{\pgfqpoint{4.269237in}{4.216001in}}%
\pgfpathlineto{\pgfqpoint{4.393252in}{4.232714in}}%
\pgfpathlineto{\pgfqpoint{4.517266in}{4.233593in}}%
\pgfpathlineto{\pgfqpoint{4.641280in}{4.255583in}}%
\pgfpathlineto{\pgfqpoint{4.765295in}{4.260860in}}%
\pgfpathlineto{\pgfqpoint{4.889309in}{4.281091in}}%
\pgfpathlineto{\pgfqpoint{5.013324in}{4.223038in}}%
\pgfpathlineto{\pgfqpoint{5.137338in}{4.160588in}}%
\pgfpathlineto{\pgfqpoint{5.261352in}{4.163227in}}%
\pgfpathlineto{\pgfqpoint{5.385367in}{4.209844in}}%
\pgfpathlineto{\pgfqpoint{5.509381in}{4.247666in}}%
\pgfpathlineto{\pgfqpoint{5.633395in}{4.268776in}}%
\pgfpathlineto{\pgfqpoint{5.757410in}{4.294284in}}%
\pgfpathlineto{\pgfqpoint{5.881424in}{4.286368in}}%
\pgfpathlineto{\pgfqpoint{6.005439in}{4.274054in}}%
\pgfpathlineto{\pgfqpoint{6.129453in}{4.270536in}}%
\pgfpathlineto{\pgfqpoint{6.253467in}{4.232714in}}%
\pgfpathlineto{\pgfqpoint{6.377482in}{4.189614in}}%
\pgfpathlineto{\pgfqpoint{6.501496in}{4.183457in}}%
\pgfpathlineto{\pgfqpoint{6.625511in}{4.199289in}}%
\pgfpathlineto{\pgfqpoint{6.749525in}{4.223038in}}%
\pgfpathlineto{\pgfqpoint{6.873539in}{4.244148in}}%
\pgfpathlineto{\pgfqpoint{6.997554in}{4.275813in}}%
\pgfusepath{stroke}%
\end{pgfscope}%
\begin{pgfscope}%
\pgfsetrectcap%
\pgfsetmiterjoin%
\pgfsetlinewidth{0.803000pt}%
\definecolor{currentstroke}{rgb}{0.000000,0.000000,0.000000}%
\pgfsetstrokecolor{currentstroke}%
\pgfsetdash{}{0pt}%
\pgfpathmoveto{\pgfqpoint{0.617014in}{0.580556in}}%
\pgfpathlineto{\pgfqpoint{0.617014in}{4.627778in}}%
\pgfusepath{stroke}%
\end{pgfscope}%
\begin{pgfscope}%
\pgfsetrectcap%
\pgfsetmiterjoin%
\pgfsetlinewidth{0.803000pt}%
\definecolor{currentstroke}{rgb}{0.000000,0.000000,0.000000}%
\pgfsetstrokecolor{currentstroke}%
\pgfsetdash{}{0pt}%
\pgfpathmoveto{\pgfqpoint{7.301389in}{0.580556in}}%
\pgfpathlineto{\pgfqpoint{7.301389in}{4.627778in}}%
\pgfusepath{stroke}%
\end{pgfscope}%
\begin{pgfscope}%
\pgfsetrectcap%
\pgfsetmiterjoin%
\pgfsetlinewidth{0.803000pt}%
\definecolor{currentstroke}{rgb}{0.000000,0.000000,0.000000}%
\pgfsetstrokecolor{currentstroke}%
\pgfsetdash{}{0pt}%
\pgfpathmoveto{\pgfqpoint{0.617014in}{0.580556in}}%
\pgfpathlineto{\pgfqpoint{7.301389in}{0.580556in}}%
\pgfusepath{stroke}%
\end{pgfscope}%
\begin{pgfscope}%
\pgfsetrectcap%
\pgfsetmiterjoin%
\pgfsetlinewidth{0.803000pt}%
\definecolor{currentstroke}{rgb}{0.000000,0.000000,0.000000}%
\pgfsetstrokecolor{currentstroke}%
\pgfsetdash{}{0pt}%
\pgfpathmoveto{\pgfqpoint{0.617014in}{4.627778in}}%
\pgfpathlineto{\pgfqpoint{7.301389in}{4.627778in}}%
\pgfusepath{stroke}%
\end{pgfscope}%
\begin{pgfscope}%
\pgftext[x=3.959201in,y=4.711111in,,base]{\sffamily\fontsize{12.000000}{14.400000}\selectfont Accelerometer Data (Hand)}%
\end{pgfscope}%
\begin{pgfscope}%
\pgfsetbuttcap%
\pgfsetmiterjoin%
\definecolor{currentfill}{rgb}{1.000000,1.000000,1.000000}%
\pgfsetfillcolor{currentfill}%
\pgfsetfillopacity{0.800000}%
\pgfsetlinewidth{1.003750pt}%
\definecolor{currentstroke}{rgb}{0.800000,0.800000,0.800000}%
\pgfsetstrokecolor{currentstroke}%
\pgfsetstrokeopacity{0.800000}%
\pgfsetdash{}{0pt}%
\pgfpathmoveto{\pgfqpoint{6.672781in}{3.906722in}}%
\pgfpathlineto{\pgfqpoint{7.204167in}{3.906722in}}%
\pgfpathquadraticcurveto{\pgfqpoint{7.231944in}{3.906722in}}{\pgfqpoint{7.231944in}{3.934500in}}%
\pgfpathlineto{\pgfqpoint{7.231944in}{4.530556in}}%
\pgfpathquadraticcurveto{\pgfqpoint{7.231944in}{4.558333in}}{\pgfqpoint{7.204167in}{4.558333in}}%
\pgfpathlineto{\pgfqpoint{6.672781in}{4.558333in}}%
\pgfpathquadraticcurveto{\pgfqpoint{6.645003in}{4.558333in}}{\pgfqpoint{6.645003in}{4.530556in}}%
\pgfpathlineto{\pgfqpoint{6.645003in}{3.934500in}}%
\pgfpathquadraticcurveto{\pgfqpoint{6.645003in}{3.906722in}}{\pgfqpoint{6.672781in}{3.906722in}}%
\pgfpathclose%
\pgfusepath{stroke,fill}%
\end{pgfscope}%
\begin{pgfscope}%
\pgfsetrectcap%
\pgfsetroundjoin%
\pgfsetlinewidth{1.505625pt}%
\definecolor{currentstroke}{rgb}{0.121569,0.466667,0.705882}%
\pgfsetstrokecolor{currentstroke}%
\pgfsetdash{}{0pt}%
\pgfpathmoveto{\pgfqpoint{6.700559in}{4.444306in}}%
\pgfpathlineto{\pgfqpoint{6.978337in}{4.444306in}}%
\pgfusepath{stroke}%
\end{pgfscope}%
\begin{pgfscope}%
\pgftext[x=7.089448in,y=4.395695in,left,base]{\sffamily\fontsize{10.000000}{12.000000}\selectfont X}%
\end{pgfscope}%
\begin{pgfscope}%
\pgfsetrectcap%
\pgfsetroundjoin%
\pgfsetlinewidth{1.505625pt}%
\definecolor{currentstroke}{rgb}{1.000000,0.498039,0.054902}%
\pgfsetstrokecolor{currentstroke}%
\pgfsetdash{}{0pt}%
\pgfpathmoveto{\pgfqpoint{6.700559in}{4.240991in}}%
\pgfpathlineto{\pgfqpoint{6.978337in}{4.240991in}}%
\pgfusepath{stroke}%
\end{pgfscope}%
\begin{pgfscope}%
\pgftext[x=7.089448in,y=4.192380in,left,base]{\sffamily\fontsize{10.000000}{12.000000}\selectfont Y}%
\end{pgfscope}%
\begin{pgfscope}%
\pgfsetrectcap%
\pgfsetroundjoin%
\pgfsetlinewidth{1.505625pt}%
\definecolor{currentstroke}{rgb}{0.172549,0.627451,0.172549}%
\pgfsetstrokecolor{currentstroke}%
\pgfsetdash{}{0pt}%
\pgfpathmoveto{\pgfqpoint{6.700559in}{4.037677in}}%
\pgfpathlineto{\pgfqpoint{6.978337in}{4.037677in}}%
\pgfusepath{stroke}%
\end{pgfscope}%
\begin{pgfscope}%
\pgftext[x=7.089448in,y=3.989066in,left,base]{\sffamily\fontsize{10.000000}{12.000000}\selectfont Z}%
\end{pgfscope}%
\end{pgfpicture}%
\makeatother%
\endgroup%
}
  \scalebox{0.1}{%% Creator: Matplotlib, PGF backend
%%
%% To include the figure in your LaTeX document, write
%%   \input{<filename>.pgf}
%%
%% Make sure the required packages are loaded in your preamble
%%   \usepackage{pgf}
%%
%% Figures using additional raster images can only be included by \input if
%% they are in the same directory as the main LaTeX file. For loading figures
%% from other directories you can use the `import` package
%%   \usepackage{import}
%% and then include the figures with
%%   \import{<path to file>}{<filename>.pgf}
%%
%% Matplotlib used the following preamble
%%   \usepackage{fontspec}
%%   \setmainfont{Times New Roman}
%%   \setsansfont{Lucida Grande}
%%   \setmonofont{Andale Mono}
%%
\begingroup%
\makeatletter%
\begin{pgfpicture}%
\pgfpathrectangle{\pgfpointorigin}{\pgfqpoint{7.500000in}{5.000000in}}%
\pgfusepath{use as bounding box, clip}%
\begin{pgfscope}%
\pgfsetbuttcap%
\pgfsetmiterjoin%
\definecolor{currentfill}{rgb}{1.000000,1.000000,1.000000}%
\pgfsetfillcolor{currentfill}%
\pgfsetlinewidth{0.000000pt}%
\definecolor{currentstroke}{rgb}{1.000000,1.000000,1.000000}%
\pgfsetstrokecolor{currentstroke}%
\pgfsetdash{}{0pt}%
\pgfpathmoveto{\pgfqpoint{0.000000in}{0.000000in}}%
\pgfpathlineto{\pgfqpoint{7.500000in}{0.000000in}}%
\pgfpathlineto{\pgfqpoint{7.500000in}{5.000000in}}%
\pgfpathlineto{\pgfqpoint{0.000000in}{5.000000in}}%
\pgfpathclose%
\pgfusepath{fill}%
\end{pgfscope}%
\begin{pgfscope}%
\pgfsetbuttcap%
\pgfsetmiterjoin%
\definecolor{currentfill}{rgb}{1.000000,1.000000,1.000000}%
\pgfsetfillcolor{currentfill}%
\pgfsetlinewidth{0.000000pt}%
\definecolor{currentstroke}{rgb}{0.000000,0.000000,0.000000}%
\pgfsetstrokecolor{currentstroke}%
\pgfsetstrokeopacity{0.000000}%
\pgfsetdash{}{0pt}%
\pgfpathmoveto{\pgfqpoint{0.884097in}{0.809444in}}%
\pgfpathlineto{\pgfqpoint{7.205428in}{0.809444in}}%
\pgfpathlineto{\pgfqpoint{7.205428in}{4.468333in}}%
\pgfpathlineto{\pgfqpoint{0.884097in}{4.468333in}}%
\pgfpathclose%
\pgfusepath{fill}%
\end{pgfscope}%
\begin{pgfscope}%
\pgfsetbuttcap%
\pgfsetroundjoin%
\definecolor{currentfill}{rgb}{0.000000,0.000000,0.000000}%
\pgfsetfillcolor{currentfill}%
\pgfsetlinewidth{0.803000pt}%
\definecolor{currentstroke}{rgb}{0.000000,0.000000,0.000000}%
\pgfsetstrokecolor{currentstroke}%
\pgfsetdash{}{0pt}%
\pgfsys@defobject{currentmarker}{\pgfqpoint{0.000000in}{-0.048611in}}{\pgfqpoint{0.000000in}{0.000000in}}{%
\pgfpathmoveto{\pgfqpoint{0.000000in}{0.000000in}}%
\pgfpathlineto{\pgfqpoint{0.000000in}{-0.048611in}}%
\pgfusepath{stroke,fill}%
}%
\begin{pgfscope}%
\pgfsys@transformshift{1.171430in}{0.809444in}%
\pgfsys@useobject{currentmarker}{}%
\end{pgfscope}%
\end{pgfscope}%
\begin{pgfscope}%
\pgftext[x=1.171430in,y=0.712222in,,top]{\sffamily\fontsize{16.000000}{19.200000}\selectfont 0}%
\end{pgfscope}%
\begin{pgfscope}%
\pgfsetbuttcap%
\pgfsetroundjoin%
\definecolor{currentfill}{rgb}{0.000000,0.000000,0.000000}%
\pgfsetfillcolor{currentfill}%
\pgfsetlinewidth{0.803000pt}%
\definecolor{currentstroke}{rgb}{0.000000,0.000000,0.000000}%
\pgfsetstrokecolor{currentstroke}%
\pgfsetdash{}{0pt}%
\pgfsys@defobject{currentmarker}{\pgfqpoint{0.000000in}{-0.048611in}}{\pgfqpoint{0.000000in}{0.000000in}}{%
\pgfpathmoveto{\pgfqpoint{0.000000in}{0.000000in}}%
\pgfpathlineto{\pgfqpoint{0.000000in}{-0.048611in}}%
\pgfusepath{stroke,fill}%
}%
\begin{pgfscope}%
\pgfsys@transformshift{2.344219in}{0.809444in}%
\pgfsys@useobject{currentmarker}{}%
\end{pgfscope}%
\end{pgfscope}%
\begin{pgfscope}%
\pgftext[x=2.344219in,y=0.712222in,,top]{\sffamily\fontsize{16.000000}{19.200000}\selectfont 100}%
\end{pgfscope}%
\begin{pgfscope}%
\pgfsetbuttcap%
\pgfsetroundjoin%
\definecolor{currentfill}{rgb}{0.000000,0.000000,0.000000}%
\pgfsetfillcolor{currentfill}%
\pgfsetlinewidth{0.803000pt}%
\definecolor{currentstroke}{rgb}{0.000000,0.000000,0.000000}%
\pgfsetstrokecolor{currentstroke}%
\pgfsetdash{}{0pt}%
\pgfsys@defobject{currentmarker}{\pgfqpoint{0.000000in}{-0.048611in}}{\pgfqpoint{0.000000in}{0.000000in}}{%
\pgfpathmoveto{\pgfqpoint{0.000000in}{0.000000in}}%
\pgfpathlineto{\pgfqpoint{0.000000in}{-0.048611in}}%
\pgfusepath{stroke,fill}%
}%
\begin{pgfscope}%
\pgfsys@transformshift{3.517008in}{0.809444in}%
\pgfsys@useobject{currentmarker}{}%
\end{pgfscope}%
\end{pgfscope}%
\begin{pgfscope}%
\pgftext[x=3.517008in,y=0.712222in,,top]{\sffamily\fontsize{16.000000}{19.200000}\selectfont 200}%
\end{pgfscope}%
\begin{pgfscope}%
\pgfsetbuttcap%
\pgfsetroundjoin%
\definecolor{currentfill}{rgb}{0.000000,0.000000,0.000000}%
\pgfsetfillcolor{currentfill}%
\pgfsetlinewidth{0.803000pt}%
\definecolor{currentstroke}{rgb}{0.000000,0.000000,0.000000}%
\pgfsetstrokecolor{currentstroke}%
\pgfsetdash{}{0pt}%
\pgfsys@defobject{currentmarker}{\pgfqpoint{0.000000in}{-0.048611in}}{\pgfqpoint{0.000000in}{0.000000in}}{%
\pgfpathmoveto{\pgfqpoint{0.000000in}{0.000000in}}%
\pgfpathlineto{\pgfqpoint{0.000000in}{-0.048611in}}%
\pgfusepath{stroke,fill}%
}%
\begin{pgfscope}%
\pgfsys@transformshift{4.689797in}{0.809444in}%
\pgfsys@useobject{currentmarker}{}%
\end{pgfscope}%
\end{pgfscope}%
\begin{pgfscope}%
\pgftext[x=4.689797in,y=0.712222in,,top]{\sffamily\fontsize{16.000000}{19.200000}\selectfont 300}%
\end{pgfscope}%
\begin{pgfscope}%
\pgfsetbuttcap%
\pgfsetroundjoin%
\definecolor{currentfill}{rgb}{0.000000,0.000000,0.000000}%
\pgfsetfillcolor{currentfill}%
\pgfsetlinewidth{0.803000pt}%
\definecolor{currentstroke}{rgb}{0.000000,0.000000,0.000000}%
\pgfsetstrokecolor{currentstroke}%
\pgfsetdash{}{0pt}%
\pgfsys@defobject{currentmarker}{\pgfqpoint{0.000000in}{-0.048611in}}{\pgfqpoint{0.000000in}{0.000000in}}{%
\pgfpathmoveto{\pgfqpoint{0.000000in}{0.000000in}}%
\pgfpathlineto{\pgfqpoint{0.000000in}{-0.048611in}}%
\pgfusepath{stroke,fill}%
}%
\begin{pgfscope}%
\pgfsys@transformshift{5.862585in}{0.809444in}%
\pgfsys@useobject{currentmarker}{}%
\end{pgfscope}%
\end{pgfscope}%
\begin{pgfscope}%
\pgftext[x=5.862585in,y=0.712222in,,top]{\sffamily\fontsize{16.000000}{19.200000}\selectfont 400}%
\end{pgfscope}%
\begin{pgfscope}%
\pgfsetbuttcap%
\pgfsetroundjoin%
\definecolor{currentfill}{rgb}{0.000000,0.000000,0.000000}%
\pgfsetfillcolor{currentfill}%
\pgfsetlinewidth{0.803000pt}%
\definecolor{currentstroke}{rgb}{0.000000,0.000000,0.000000}%
\pgfsetstrokecolor{currentstroke}%
\pgfsetdash{}{0pt}%
\pgfsys@defobject{currentmarker}{\pgfqpoint{0.000000in}{-0.048611in}}{\pgfqpoint{0.000000in}{0.000000in}}{%
\pgfpathmoveto{\pgfqpoint{0.000000in}{0.000000in}}%
\pgfpathlineto{\pgfqpoint{0.000000in}{-0.048611in}}%
\pgfusepath{stroke,fill}%
}%
\begin{pgfscope}%
\pgfsys@transformshift{7.035374in}{0.809444in}%
\pgfsys@useobject{currentmarker}{}%
\end{pgfscope}%
\end{pgfscope}%
\begin{pgfscope}%
\pgftext[x=7.035374in,y=0.712222in,,top]{\sffamily\fontsize{16.000000}{19.200000}\selectfont 500}%
\end{pgfscope}%
\begin{pgfscope}%
\pgftext[x=4.044763in,y=0.442474in,,top]{\sffamily\fontsize{16.000000}{19.200000}\selectfont Time (ms)}%
\end{pgfscope}%
\begin{pgfscope}%
\pgfsetbuttcap%
\pgfsetroundjoin%
\definecolor{currentfill}{rgb}{0.000000,0.000000,0.000000}%
\pgfsetfillcolor{currentfill}%
\pgfsetlinewidth{0.803000pt}%
\definecolor{currentstroke}{rgb}{0.000000,0.000000,0.000000}%
\pgfsetstrokecolor{currentstroke}%
\pgfsetdash{}{0pt}%
\pgfsys@defobject{currentmarker}{\pgfqpoint{-0.048611in}{0.000000in}}{\pgfqpoint{0.000000in}{0.000000in}}{%
\pgfpathmoveto{\pgfqpoint{0.000000in}{0.000000in}}%
\pgfpathlineto{\pgfqpoint{-0.048611in}{0.000000in}}%
\pgfusepath{stroke,fill}%
}%
\begin{pgfscope}%
\pgfsys@transformshift{0.884097in}{0.815254in}%
\pgfsys@useobject{currentmarker}{}%
\end{pgfscope}%
\end{pgfscope}%
\begin{pgfscope}%
\pgftext[x=0.646358in,y=0.729588in,left,base]{\sffamily\fontsize{16.000000}{19.200000}\selectfont 0}%
\end{pgfscope}%
\begin{pgfscope}%
\pgfsetbuttcap%
\pgfsetroundjoin%
\definecolor{currentfill}{rgb}{0.000000,0.000000,0.000000}%
\pgfsetfillcolor{currentfill}%
\pgfsetlinewidth{0.803000pt}%
\definecolor{currentstroke}{rgb}{0.000000,0.000000,0.000000}%
\pgfsetstrokecolor{currentstroke}%
\pgfsetdash{}{0pt}%
\pgfsys@defobject{currentmarker}{\pgfqpoint{-0.048611in}{0.000000in}}{\pgfqpoint{0.000000in}{0.000000in}}{%
\pgfpathmoveto{\pgfqpoint{0.000000in}{0.000000in}}%
\pgfpathlineto{\pgfqpoint{-0.048611in}{0.000000in}}%
\pgfusepath{stroke,fill}%
}%
\begin{pgfscope}%
\pgfsys@transformshift{0.884097in}{1.513332in}%
\pgfsys@useobject{currentmarker}{}%
\end{pgfscope}%
\end{pgfscope}%
\begin{pgfscope}%
\pgftext[x=0.646358in,y=1.427665in,left,base]{\sffamily\fontsize{16.000000}{19.200000}\selectfont 2}%
\end{pgfscope}%
\begin{pgfscope}%
\pgfsetbuttcap%
\pgfsetroundjoin%
\definecolor{currentfill}{rgb}{0.000000,0.000000,0.000000}%
\pgfsetfillcolor{currentfill}%
\pgfsetlinewidth{0.803000pt}%
\definecolor{currentstroke}{rgb}{0.000000,0.000000,0.000000}%
\pgfsetstrokecolor{currentstroke}%
\pgfsetdash{}{0pt}%
\pgfsys@defobject{currentmarker}{\pgfqpoint{-0.048611in}{0.000000in}}{\pgfqpoint{0.000000in}{0.000000in}}{%
\pgfpathmoveto{\pgfqpoint{0.000000in}{0.000000in}}%
\pgfpathlineto{\pgfqpoint{-0.048611in}{0.000000in}}%
\pgfusepath{stroke,fill}%
}%
\begin{pgfscope}%
\pgfsys@transformshift{0.884097in}{2.211409in}%
\pgfsys@useobject{currentmarker}{}%
\end{pgfscope}%
\end{pgfscope}%
\begin{pgfscope}%
\pgftext[x=0.646358in,y=2.125743in,left,base]{\sffamily\fontsize{16.000000}{19.200000}\selectfont 4}%
\end{pgfscope}%
\begin{pgfscope}%
\pgfsetbuttcap%
\pgfsetroundjoin%
\definecolor{currentfill}{rgb}{0.000000,0.000000,0.000000}%
\pgfsetfillcolor{currentfill}%
\pgfsetlinewidth{0.803000pt}%
\definecolor{currentstroke}{rgb}{0.000000,0.000000,0.000000}%
\pgfsetstrokecolor{currentstroke}%
\pgfsetdash{}{0pt}%
\pgfsys@defobject{currentmarker}{\pgfqpoint{-0.048611in}{0.000000in}}{\pgfqpoint{0.000000in}{0.000000in}}{%
\pgfpathmoveto{\pgfqpoint{0.000000in}{0.000000in}}%
\pgfpathlineto{\pgfqpoint{-0.048611in}{0.000000in}}%
\pgfusepath{stroke,fill}%
}%
\begin{pgfscope}%
\pgfsys@transformshift{0.884097in}{2.909486in}%
\pgfsys@useobject{currentmarker}{}%
\end{pgfscope}%
\end{pgfscope}%
\begin{pgfscope}%
\pgftext[x=0.646358in,y=2.823820in,left,base]{\sffamily\fontsize{16.000000}{19.200000}\selectfont 6}%
\end{pgfscope}%
\begin{pgfscope}%
\pgfsetbuttcap%
\pgfsetroundjoin%
\definecolor{currentfill}{rgb}{0.000000,0.000000,0.000000}%
\pgfsetfillcolor{currentfill}%
\pgfsetlinewidth{0.803000pt}%
\definecolor{currentstroke}{rgb}{0.000000,0.000000,0.000000}%
\pgfsetstrokecolor{currentstroke}%
\pgfsetdash{}{0pt}%
\pgfsys@defobject{currentmarker}{\pgfqpoint{-0.048611in}{0.000000in}}{\pgfqpoint{0.000000in}{0.000000in}}{%
\pgfpathmoveto{\pgfqpoint{0.000000in}{0.000000in}}%
\pgfpathlineto{\pgfqpoint{-0.048611in}{0.000000in}}%
\pgfusepath{stroke,fill}%
}%
\begin{pgfscope}%
\pgfsys@transformshift{0.884097in}{3.607564in}%
\pgfsys@useobject{currentmarker}{}%
\end{pgfscope}%
\end{pgfscope}%
\begin{pgfscope}%
\pgftext[x=0.646358in,y=3.521897in,left,base]{\sffamily\fontsize{16.000000}{19.200000}\selectfont 8}%
\end{pgfscope}%
\begin{pgfscope}%
\pgfsetbuttcap%
\pgfsetroundjoin%
\definecolor{currentfill}{rgb}{0.000000,0.000000,0.000000}%
\pgfsetfillcolor{currentfill}%
\pgfsetlinewidth{0.803000pt}%
\definecolor{currentstroke}{rgb}{0.000000,0.000000,0.000000}%
\pgfsetstrokecolor{currentstroke}%
\pgfsetdash{}{0pt}%
\pgfsys@defobject{currentmarker}{\pgfqpoint{-0.048611in}{0.000000in}}{\pgfqpoint{0.000000in}{0.000000in}}{%
\pgfpathmoveto{\pgfqpoint{0.000000in}{0.000000in}}%
\pgfpathlineto{\pgfqpoint{-0.048611in}{0.000000in}}%
\pgfusepath{stroke,fill}%
}%
\begin{pgfscope}%
\pgfsys@transformshift{0.884097in}{4.305641in}%
\pgfsys@useobject{currentmarker}{}%
\end{pgfscope}%
\end{pgfscope}%
\begin{pgfscope}%
\pgftext[x=0.505842in,y=4.219975in,left,base]{\sffamily\fontsize{16.000000}{19.200000}\selectfont 10}%
\end{pgfscope}%
\begin{pgfscope}%
\pgftext[x=0.450286in,y=2.638889in,,bottom,rotate=90.000000]{\sffamily\fontsize{16.000000}{19.200000}\selectfont Acceleration (m/s\^2)}%
\end{pgfscope}%
\begin{pgfscope}%
\pgfpathrectangle{\pgfqpoint{0.884097in}{0.809444in}}{\pgfqpoint{6.321331in}{3.658889in}} %
\pgfusepath{clip}%
\pgfsetrectcap%
\pgfsetroundjoin%
\pgfsetlinewidth{1.505625pt}%
\definecolor{currentstroke}{rgb}{0.121569,0.466667,0.705882}%
\pgfsetstrokecolor{currentstroke}%
\pgfsetdash{}{0pt}%
\pgfpathmoveto{\pgfqpoint{1.171430in}{1.007524in}}%
\pgfpathlineto{\pgfqpoint{1.288709in}{1.003344in}}%
\pgfpathlineto{\pgfqpoint{1.405988in}{1.001672in}}%
\pgfpathlineto{\pgfqpoint{1.523267in}{1.001672in}}%
\pgfpathlineto{\pgfqpoint{1.640546in}{1.008360in}}%
\pgfpathlineto{\pgfqpoint{1.757825in}{1.011704in}}%
\pgfpathlineto{\pgfqpoint{1.875104in}{1.010868in}}%
\pgfpathlineto{\pgfqpoint{1.992383in}{1.007524in}}%
\pgfpathlineto{\pgfqpoint{2.109661in}{1.009196in}}%
\pgfpathlineto{\pgfqpoint{2.226940in}{1.000836in}}%
\pgfpathlineto{\pgfqpoint{2.344219in}{1.000836in}}%
\pgfpathlineto{\pgfqpoint{2.461498in}{1.002508in}}%
\pgfpathlineto{\pgfqpoint{2.578777in}{1.001672in}}%
\pgfpathlineto{\pgfqpoint{2.696056in}{1.010868in}}%
\pgfpathlineto{\pgfqpoint{2.813335in}{0.997492in}}%
\pgfpathlineto{\pgfqpoint{2.930614in}{1.001672in}}%
\pgfpathlineto{\pgfqpoint{3.047892in}{1.003344in}}%
\pgfpathlineto{\pgfqpoint{3.165171in}{1.008360in}}%
\pgfpathlineto{\pgfqpoint{3.282450in}{1.005852in}}%
\pgfpathlineto{\pgfqpoint{3.399729in}{1.007524in}}%
\pgfpathlineto{\pgfqpoint{3.517008in}{1.003344in}}%
\pgfpathlineto{\pgfqpoint{3.634287in}{1.005852in}}%
\pgfpathlineto{\pgfqpoint{3.751566in}{1.004180in}}%
\pgfpathlineto{\pgfqpoint{3.868845in}{1.002508in}}%
\pgfpathlineto{\pgfqpoint{3.986123in}{1.013376in}}%
\pgfpathlineto{\pgfqpoint{4.103402in}{1.003344in}}%
\pgfpathlineto{\pgfqpoint{4.220681in}{1.004180in}}%
\pgfpathlineto{\pgfqpoint{4.337960in}{1.008360in}}%
\pgfpathlineto{\pgfqpoint{4.455239in}{1.009196in}}%
\pgfpathlineto{\pgfqpoint{4.572518in}{1.004180in}}%
\pgfpathlineto{\pgfqpoint{4.689797in}{1.001672in}}%
\pgfpathlineto{\pgfqpoint{4.807076in}{1.001672in}}%
\pgfpathlineto{\pgfqpoint{4.924354in}{1.005852in}}%
\pgfpathlineto{\pgfqpoint{5.041633in}{1.000000in}}%
\pgfpathlineto{\pgfqpoint{5.158912in}{0.999164in}}%
\pgfpathlineto{\pgfqpoint{5.276191in}{1.002508in}}%
\pgfpathlineto{\pgfqpoint{5.393470in}{1.004180in}}%
\pgfpathlineto{\pgfqpoint{5.510749in}{1.007524in}}%
\pgfpathlineto{\pgfqpoint{5.628028in}{1.007524in}}%
\pgfpathlineto{\pgfqpoint{5.745306in}{1.005016in}}%
\pgfpathlineto{\pgfqpoint{5.862585in}{1.001672in}}%
\pgfpathlineto{\pgfqpoint{5.979864in}{1.003344in}}%
\pgfpathlineto{\pgfqpoint{6.097143in}{1.001672in}}%
\pgfpathlineto{\pgfqpoint{6.214422in}{0.998328in}}%
\pgfpathlineto{\pgfqpoint{6.331701in}{1.005016in}}%
\pgfpathlineto{\pgfqpoint{6.448980in}{1.001672in}}%
\pgfpathlineto{\pgfqpoint{6.566259in}{1.005852in}}%
\pgfpathlineto{\pgfqpoint{6.683537in}{1.007524in}}%
\pgfpathlineto{\pgfqpoint{6.800816in}{1.005016in}}%
\pgfpathlineto{\pgfqpoint{6.918095in}{1.001672in}}%
\pgfusepath{stroke}%
\end{pgfscope}%
\begin{pgfscope}%
\pgfpathrectangle{\pgfqpoint{0.884097in}{0.809444in}}{\pgfqpoint{6.321331in}{3.658889in}} %
\pgfusepath{clip}%
\pgfsetrectcap%
\pgfsetroundjoin%
\pgfsetlinewidth{1.505625pt}%
\definecolor{currentstroke}{rgb}{1.000000,0.498039,0.054902}%
\pgfsetstrokecolor{currentstroke}%
\pgfsetdash{}{0pt}%
\pgfpathmoveto{\pgfqpoint{1.171430in}{0.979937in}}%
\pgfpathlineto{\pgfqpoint{1.288709in}{0.989133in}}%
\pgfpathlineto{\pgfqpoint{1.405988in}{0.983281in}}%
\pgfpathlineto{\pgfqpoint{1.523267in}{0.989969in}}%
\pgfpathlineto{\pgfqpoint{1.640546in}{0.981609in}}%
\pgfpathlineto{\pgfqpoint{1.757825in}{0.987461in}}%
\pgfpathlineto{\pgfqpoint{1.875104in}{0.998328in}}%
\pgfpathlineto{\pgfqpoint{1.992383in}{0.989969in}}%
\pgfpathlineto{\pgfqpoint{2.109661in}{0.981609in}}%
\pgfpathlineto{\pgfqpoint{2.226940in}{0.986625in}}%
\pgfpathlineto{\pgfqpoint{2.344219in}{0.988297in}}%
\pgfpathlineto{\pgfqpoint{2.461498in}{0.984117in}}%
\pgfpathlineto{\pgfqpoint{2.578777in}{0.988297in}}%
\pgfpathlineto{\pgfqpoint{2.696056in}{0.993313in}}%
\pgfpathlineto{\pgfqpoint{2.813335in}{0.983281in}}%
\pgfpathlineto{\pgfqpoint{2.930614in}{0.982445in}}%
\pgfpathlineto{\pgfqpoint{3.047892in}{0.982445in}}%
\pgfpathlineto{\pgfqpoint{3.165171in}{0.982445in}}%
\pgfpathlineto{\pgfqpoint{3.282450in}{0.990805in}}%
\pgfpathlineto{\pgfqpoint{3.399729in}{0.986625in}}%
\pgfpathlineto{\pgfqpoint{3.517008in}{0.983281in}}%
\pgfpathlineto{\pgfqpoint{3.634287in}{0.980773in}}%
\pgfpathlineto{\pgfqpoint{3.751566in}{0.986625in}}%
\pgfpathlineto{\pgfqpoint{3.868845in}{0.984953in}}%
\pgfpathlineto{\pgfqpoint{3.986123in}{0.985789in}}%
\pgfpathlineto{\pgfqpoint{4.103402in}{0.984117in}}%
\pgfpathlineto{\pgfqpoint{4.220681in}{0.986625in}}%
\pgfpathlineto{\pgfqpoint{4.337960in}{0.984117in}}%
\pgfpathlineto{\pgfqpoint{4.455239in}{0.984953in}}%
\pgfpathlineto{\pgfqpoint{4.572518in}{0.981609in}}%
\pgfpathlineto{\pgfqpoint{4.689797in}{0.983281in}}%
\pgfpathlineto{\pgfqpoint{4.807076in}{0.975758in}}%
\pgfpathlineto{\pgfqpoint{4.924354in}{0.984953in}}%
\pgfpathlineto{\pgfqpoint{5.041633in}{0.982445in}}%
\pgfpathlineto{\pgfqpoint{5.158912in}{0.981609in}}%
\pgfpathlineto{\pgfqpoint{5.276191in}{0.989969in}}%
\pgfpathlineto{\pgfqpoint{5.393470in}{0.991641in}}%
\pgfpathlineto{\pgfqpoint{5.510749in}{0.986625in}}%
\pgfpathlineto{\pgfqpoint{5.628028in}{0.985789in}}%
\pgfpathlineto{\pgfqpoint{5.745306in}{0.979937in}}%
\pgfpathlineto{\pgfqpoint{5.862585in}{0.987461in}}%
\pgfpathlineto{\pgfqpoint{5.979864in}{0.990805in}}%
\pgfpathlineto{\pgfqpoint{6.097143in}{0.980773in}}%
\pgfpathlineto{\pgfqpoint{6.214422in}{0.983281in}}%
\pgfpathlineto{\pgfqpoint{6.331701in}{0.991641in}}%
\pgfpathlineto{\pgfqpoint{6.448980in}{0.991641in}}%
\pgfpathlineto{\pgfqpoint{6.566259in}{0.985789in}}%
\pgfpathlineto{\pgfqpoint{6.683537in}{0.992477in}}%
\pgfpathlineto{\pgfqpoint{6.800816in}{0.995820in}}%
\pgfpathlineto{\pgfqpoint{6.918095in}{0.985789in}}%
\pgfusepath{stroke}%
\end{pgfscope}%
\begin{pgfscope}%
\pgfpathrectangle{\pgfqpoint{0.884097in}{0.809444in}}{\pgfqpoint{6.321331in}{3.658889in}} %
\pgfusepath{clip}%
\pgfsetrectcap%
\pgfsetroundjoin%
\pgfsetlinewidth{1.505625pt}%
\definecolor{currentstroke}{rgb}{0.172549,0.627451,0.172549}%
\pgfsetstrokecolor{currentstroke}%
\pgfsetdash{}{0pt}%
\pgfpathmoveto{\pgfqpoint{1.171430in}{4.283629in}}%
\pgfpathlineto{\pgfqpoint{1.288709in}{4.287809in}}%
\pgfpathlineto{\pgfqpoint{1.405988in}{4.281121in}}%
\pgfpathlineto{\pgfqpoint{1.523267in}{4.291989in}}%
\pgfpathlineto{\pgfqpoint{1.640546in}{4.289481in}}%
\pgfpathlineto{\pgfqpoint{1.757825in}{4.287809in}}%
\pgfpathlineto{\pgfqpoint{1.875104in}{4.287809in}}%
\pgfpathlineto{\pgfqpoint{1.992383in}{4.299512in}}%
\pgfpathlineto{\pgfqpoint{2.109661in}{4.294497in}}%
\pgfpathlineto{\pgfqpoint{2.226940in}{4.284465in}}%
\pgfpathlineto{\pgfqpoint{2.344219in}{4.287809in}}%
\pgfpathlineto{\pgfqpoint{2.461498in}{4.302020in}}%
\pgfpathlineto{\pgfqpoint{2.578777in}{4.286973in}}%
\pgfpathlineto{\pgfqpoint{2.696056in}{4.292825in}}%
\pgfpathlineto{\pgfqpoint{2.813335in}{4.300348in}}%
\pgfpathlineto{\pgfqpoint{2.930614in}{4.292825in}}%
\pgfpathlineto{\pgfqpoint{3.047892in}{4.293661in}}%
\pgfpathlineto{\pgfqpoint{3.165171in}{4.298677in}}%
\pgfpathlineto{\pgfqpoint{3.282450in}{4.297005in}}%
\pgfpathlineto{\pgfqpoint{3.399729in}{4.291153in}}%
\pgfpathlineto{\pgfqpoint{3.517008in}{4.286973in}}%
\pgfpathlineto{\pgfqpoint{3.634287in}{4.294497in}}%
\pgfpathlineto{\pgfqpoint{3.751566in}{4.290317in}}%
\pgfpathlineto{\pgfqpoint{3.868845in}{4.286973in}}%
\pgfpathlineto{\pgfqpoint{3.986123in}{4.285301in}}%
\pgfpathlineto{\pgfqpoint{4.103402in}{4.292825in}}%
\pgfpathlineto{\pgfqpoint{4.220681in}{4.297005in}}%
\pgfpathlineto{\pgfqpoint{4.337960in}{4.291989in}}%
\pgfpathlineto{\pgfqpoint{4.455239in}{4.288645in}}%
\pgfpathlineto{\pgfqpoint{4.572518in}{4.284465in}}%
\pgfpathlineto{\pgfqpoint{4.689797in}{4.293661in}}%
\pgfpathlineto{\pgfqpoint{4.807076in}{4.292825in}}%
\pgfpathlineto{\pgfqpoint{4.924354in}{4.290317in}}%
\pgfpathlineto{\pgfqpoint{5.041633in}{4.286973in}}%
\pgfpathlineto{\pgfqpoint{5.158912in}{4.282793in}}%
\pgfpathlineto{\pgfqpoint{5.276191in}{4.286137in}}%
\pgfpathlineto{\pgfqpoint{5.393470in}{4.289481in}}%
\pgfpathlineto{\pgfqpoint{5.510749in}{4.279449in}}%
\pgfpathlineto{\pgfqpoint{5.628028in}{4.287809in}}%
\pgfpathlineto{\pgfqpoint{5.745306in}{4.290317in}}%
\pgfpathlineto{\pgfqpoint{5.862585in}{4.295333in}}%
\pgfpathlineto{\pgfqpoint{5.979864in}{4.293661in}}%
\pgfpathlineto{\pgfqpoint{6.097143in}{4.287809in}}%
\pgfpathlineto{\pgfqpoint{6.214422in}{4.281121in}}%
\pgfpathlineto{\pgfqpoint{6.331701in}{4.282793in}}%
\pgfpathlineto{\pgfqpoint{6.448980in}{4.287809in}}%
\pgfpathlineto{\pgfqpoint{6.566259in}{4.295333in}}%
\pgfpathlineto{\pgfqpoint{6.683537in}{4.293661in}}%
\pgfpathlineto{\pgfqpoint{6.800816in}{4.290317in}}%
\pgfpathlineto{\pgfqpoint{6.918095in}{4.286973in}}%
\pgfusepath{stroke}%
\end{pgfscope}%
\begin{pgfscope}%
\pgfsetrectcap%
\pgfsetmiterjoin%
\pgfsetlinewidth{0.803000pt}%
\definecolor{currentstroke}{rgb}{0.000000,0.000000,0.000000}%
\pgfsetstrokecolor{currentstroke}%
\pgfsetdash{}{0pt}%
\pgfpathmoveto{\pgfqpoint{0.884097in}{0.809444in}}%
\pgfpathlineto{\pgfqpoint{0.884097in}{4.468333in}}%
\pgfusepath{stroke}%
\end{pgfscope}%
\begin{pgfscope}%
\pgfsetrectcap%
\pgfsetmiterjoin%
\pgfsetlinewidth{0.803000pt}%
\definecolor{currentstroke}{rgb}{0.000000,0.000000,0.000000}%
\pgfsetstrokecolor{currentstroke}%
\pgfsetdash{}{0pt}%
\pgfpathmoveto{\pgfqpoint{7.205428in}{0.809444in}}%
\pgfpathlineto{\pgfqpoint{7.205428in}{4.468333in}}%
\pgfusepath{stroke}%
\end{pgfscope}%
\begin{pgfscope}%
\pgfsetrectcap%
\pgfsetmiterjoin%
\pgfsetlinewidth{0.803000pt}%
\definecolor{currentstroke}{rgb}{0.000000,0.000000,0.000000}%
\pgfsetstrokecolor{currentstroke}%
\pgfsetdash{}{0pt}%
\pgfpathmoveto{\pgfqpoint{0.884097in}{0.809444in}}%
\pgfpathlineto{\pgfqpoint{7.205428in}{0.809444in}}%
\pgfusepath{stroke}%
\end{pgfscope}%
\begin{pgfscope}%
\pgfsetrectcap%
\pgfsetmiterjoin%
\pgfsetlinewidth{0.803000pt}%
\definecolor{currentstroke}{rgb}{0.000000,0.000000,0.000000}%
\pgfsetstrokecolor{currentstroke}%
\pgfsetdash{}{0pt}%
\pgfpathmoveto{\pgfqpoint{0.884097in}{4.468333in}}%
\pgfpathlineto{\pgfqpoint{7.205428in}{4.468333in}}%
\pgfusepath{stroke}%
\end{pgfscope}%
\begin{pgfscope}%
\pgftext[x=4.044763in,y=4.551667in,,base]{\sffamily\fontsize{19.200000}{23.040000}\selectfont Accelerometer Data (Table)}%
\end{pgfscope}%
\begin{pgfscope}%
\pgfsetbuttcap%
\pgfsetmiterjoin%
\definecolor{currentfill}{rgb}{1.000000,1.000000,1.000000}%
\pgfsetfillcolor{currentfill}%
\pgfsetfillopacity{0.800000}%
\pgfsetlinewidth{1.003750pt}%
\definecolor{currentstroke}{rgb}{0.800000,0.800000,0.800000}%
\pgfsetstrokecolor{currentstroke}%
\pgfsetstrokeopacity{0.800000}%
\pgfsetdash{}{0pt}%
\pgfpathmoveto{\pgfqpoint{6.199656in}{2.117600in}}%
\pgfpathlineto{\pgfqpoint{7.049873in}{2.117600in}}%
\pgfpathquadraticcurveto{\pgfqpoint{7.094317in}{2.117600in}}{\pgfqpoint{7.094317in}{2.162045in}}%
\pgfpathlineto{\pgfqpoint{7.094317in}{3.115733in}}%
\pgfpathquadraticcurveto{\pgfqpoint{7.094317in}{3.160178in}}{\pgfqpoint{7.049873in}{3.160178in}}%
\pgfpathlineto{\pgfqpoint{6.199656in}{3.160178in}}%
\pgfpathquadraticcurveto{\pgfqpoint{6.155211in}{3.160178in}}{\pgfqpoint{6.155211in}{3.115733in}}%
\pgfpathlineto{\pgfqpoint{6.155211in}{2.162045in}}%
\pgfpathquadraticcurveto{\pgfqpoint{6.155211in}{2.117600in}}{\pgfqpoint{6.199656in}{2.117600in}}%
\pgfpathclose%
\pgfusepath{stroke,fill}%
\end{pgfscope}%
\begin{pgfscope}%
\pgfsetrectcap%
\pgfsetroundjoin%
\pgfsetlinewidth{1.505625pt}%
\definecolor{currentstroke}{rgb}{0.121569,0.466667,0.705882}%
\pgfsetstrokecolor{currentstroke}%
\pgfsetdash{}{0pt}%
\pgfpathmoveto{\pgfqpoint{6.244100in}{2.977734in}}%
\pgfpathlineto{\pgfqpoint{6.688545in}{2.977734in}}%
\pgfusepath{stroke}%
\end{pgfscope}%
\begin{pgfscope}%
\pgftext[x=6.866322in,y=2.899956in,left,base]{\sffamily\fontsize{16.000000}{19.200000}\selectfont X}%
\end{pgfscope}%
\begin{pgfscope}%
\pgfsetrectcap%
\pgfsetroundjoin%
\pgfsetlinewidth{1.505625pt}%
\definecolor{currentstroke}{rgb}{1.000000,0.498039,0.054902}%
\pgfsetstrokecolor{currentstroke}%
\pgfsetdash{}{0pt}%
\pgfpathmoveto{\pgfqpoint{6.244100in}{2.652430in}}%
\pgfpathlineto{\pgfqpoint{6.688545in}{2.652430in}}%
\pgfusepath{stroke}%
\end{pgfscope}%
\begin{pgfscope}%
\pgftext[x=6.866322in,y=2.574653in,left,base]{\sffamily\fontsize{16.000000}{19.200000}\selectfont Y}%
\end{pgfscope}%
\begin{pgfscope}%
\pgfsetrectcap%
\pgfsetroundjoin%
\pgfsetlinewidth{1.505625pt}%
\definecolor{currentstroke}{rgb}{0.172549,0.627451,0.172549}%
\pgfsetstrokecolor{currentstroke}%
\pgfsetdash{}{0pt}%
\pgfpathmoveto{\pgfqpoint{6.244100in}{2.327127in}}%
\pgfpathlineto{\pgfqpoint{6.688545in}{2.327127in}}%
\pgfusepath{stroke}%
\end{pgfscope}%
\begin{pgfscope}%
\pgftext[x=6.866322in,y=2.249349in,left,base]{\sffamily\fontsize{16.000000}{19.200000}\selectfont Z}%
\end{pgfscope}%
\end{pgfpicture}%
\makeatother%
\endgroup%
}
  \caption{Typical acceleration graphs for our four states for a Nexus 5X.}
  \label{fig:AccelDiffStates}
\end{center}
\end{figure}

%\begin{figure}[t]
%\center
%\includegraphics[scale=0.25]{won_table}
%\includegraphics[scale=0.25]{joanna_table}
%\caption{Graphs of the acceleration of the Nexus 5 and 5X while on a table (at different times).}
%\end{figure}

Our goal is to allow the classifier to predict phone states no matter what action the user may be performing.
This includes times when the phone is not physically on the user, including cases that may be very difficult to differentiate: a phone being in a still backpack versus on a table, for example.
Later, we propose a potential solution to this problem. 


\subsection{Features}
For creating features, the relevant sensor data were the accelerometer readings (X, Y, Z values), number of unlocks, number of screen touches, and number of times the screen turned on/off. 
For each window of 0.5s of these raw sensor readings, we generate the following features:

\begin{enumerate}
\item \textit{Total number of phone unlocks}
\item \textit{Total number of phone touches}
\item \textit{Fraction of window that phone screen was on}
\item \textit{Mean acceleration in each of X, Y, Z}
\item \textit{Std. deviation of acceleration in each of X, Y, Z}
\item \textit{Mean magnitude of acceleration in each of X, Y, Z}
\item \textit{Std. deviation of magnitude of acceleration in each of X, Y, Z}
\item \textit{Phone is flat (handcrafted feature explained below)}
\end{enumerate}

The ``Phone is flat'' feature is a boolean feature derived from the raw accelerometer readings. 
The feature is 1 if the three equations below all hold, and 0 otherwise.
\begin{align*}
 \text{(Mean X Accel. Magnitude)} &< 1.0\\
\text{(Mean Y Accel. Magnitude)} &< 1.0\\
|9.8 - \text{(Mean Z Accel. Magnitude)}| &< 1.0
\end{align*}

Other sensors that we considered to be relevant for predicting phone states are the batched light sensor and step count.
However, the batched light sensor was not used because of its inability to distinguish outdoor nighttime darkness and the darkness from an enclosed backpack. 
We could not use the step count sensor because the sensor data we collected showed that this sensor was not reliable for our phones. 

We do not utilize a overlapping or `rolling' window.  Instead, we take distinct chunks of 0.5 seconds.
We believe that this is acceptable since the window size is small enough that we would not miss most transitions.
This decision is supported by previous work that found no notable difference in accuracy between using overlapping and non-overlapping windows \cite{Martin2013}.

\subsection{Architecture}
Our architecture has two parts.
The first consists of convolutional layers, while the second part contains dense fully connected layers.


\begin{figure*}[!h]
  \vspace{-0.2cm}
  \centering
   {\epsfig{file = convnet1, width = \linewidth}}
  \caption{The architecture of our convolutional neural net}
  \label{fig:ConvNet}
  \vspace{-0.1cm}
\end{figure*}


\begin{table}[!h]
\begin{center}
\begin{tabular}{llrp{2.5cm}}\toprule
Layer 	&  	 	Filters 	& 	Outputs  	&  	Activation \newline / Note\\\midrule
Conv1D  	&  	64 		& 	$48 \times 3$	&  ReLU \\
Conv1D  & 	64 		&	$46 \times 3 $ 	& ReLU  \\
MaxPooling1D  &  64 		& 	$15 \times 3$	& Stride = 3\\
Conv1D & 		128 	& 	$13 \times 3$&  ReLU\\
Conv1D & 128 & $11 \times 3$ & ReLU\\
GAP1D & 128 & $128 \times 1$ & \\
Dropout & | &  $128 \times 1$ & Rate = 0.5\\
Concat  & | & $144 \times 1$&  ($128 \times 1$) \newline + ($16 \times 1$)\\
Dense & | & 64 & ReLU; 6 of these in succession \\
Dense & | & 4 & Softmax

\end{tabular}
\caption{Model Architecture. For all convolutional layers, the kernel was $3 \times 6$, and GAP1D is short for GlobalAveragePooling1D. In the concatenation layer, the count features ($16 \times 1$) are concatenated with convolutional outputs to form the initial input into the dense layers.}
\label{tab:ArchDescription}
\end{center}
\end{table}

In the convolution section, we use the raw acceleration data, which includes the acceleration in the x, y, z directions. 
After multiple one-dimensional convolution layers, max and global pooling, and dropout layers, 
we concatenate the 16 features above in order to incorporate the features that do not involve the data from the accelerometer.  
Together, these features are fed into the second part of the net.  
We chose to separate the features in this way in order to take advantage of the potentially periodic behavior of a user's acceleration in certain positions (e.g. walking).
The features that use the acceleration in the X, Y, Z are separated from the other binary features because we wanted to capture the time series data of the different phone states.
The other binary features are independent of the time, so these features are added in after the acceleration features go through the convolution layers.  

After the concatenation of inputs, our model has 6 dense layers culminating in 4 outputs, which match the four classes listed previously. 
Our model is shown in Figure ~\ref{fig:ConvNet} and a description is shown in Table ~\ref{tab:ArchDescription}.
We have experimented with other architecture, such as separate binary linear classifiers for each phone state and separate neural net classifiers for each phone state.
However, we found out this multiclass neural net classifier works best and has the highest accuracy rates. 
We also experimented with the number of convolutional and dense layers as well as the number of layer units and width of the convolutional filters.

\section{Evaluation}

\subsection{Data Collection Tools}
\subsubsection{Data Collection Software}
The software used to collect data for this experiment is the AppMon logging mobile application 
for the Android phone. 
The AppMon application logs the smartphone sensors, including the following:
\begin{enumerate}
\item Accelerometer
\item Number of unlocks
\item Number of screen touches
\item Number of times the screen turns on and off
\end{enumerate}

In our project, we will be using data from the sensors
to create features in order to classify and predict the location of the phone. 

The app collects accelerometer readings (X, Y, Z directions) in units of $m/s^2$ every 10ms.
It also registers the timestamp at which various trigger events occur (e.g. phone unlocks, screen on, etc.)

\subsubsection{Phones}
We used two phones for the purpose of diary studies and collecting data for creating the classifiers.
One phone was a Nexus 5 and the other was a Neuxs 5s


\subsection {Diary Study}
To obtain data for training and validation, one participant used the test phone with the app installed 
and underwent a daily routine.
Whenever a change of state occured [Note: As of this point, would the reader know what the states are?], the participant would record the time (according to the phone) and the new state.


The distribution of the states during the diary study are shown below:


\begin{center}
\begin{tabular}{ |c|c| } 
 \hline
 Table & 0.507 \\ 
 Backpack & 0.295 \\ 
 Hand & 0.070 \\ 
 Pocket & 0.128 \\
 \hline
\end{tabular}
\end{center}
[Enter distribution stuff here @Steven]
\section{Results}
\subsection{Single Phone Model}
Using the architecture detailed in Section 3, we trained the network for 10 epochs
and measured its effectiveness using $k$-fold cross-validation with $k = 10$.
Training/validation data was compiled from the four different diary studies
that used the Nexus 5X.

In normal $k$-fold cross validation, the dataset is randomly partitioned into
$k$-folds. However, we wanted to ensure that for each fold, the validation set
was not biased towards the training set. Specifically, for any state, two samples collected at sequential timesteps
(e.g. sample A at 11:30:00 and sample B at 11:30:30 while the phone was in 
the user's pocket) are likely to be very similar. Then, if one sample 
were partitioned into the training set and the other sample into the validation set, we suspected
that it could be the case that we observe a high validation accuracy, but instead of learning 
a generalizable decision rule, the network may have only learned to remember the class of the sample in the training set 
and regurgitate that class when seeing the very similar other sample in the validation set. 
Our suspicions were confirmed when we observed that the cross-validation accuracy
was $9.22\%$ higher when the dataset was partitioned randomly instead of sequentially.

Accordingly, we decided to partition the data sequentially (i.e. without randomization)
for cross-validation. However, since the diary study data for the Nexus 5X
did not have equal proportions of each class, it was possible for some folds
to have a training set with barely any instances of a class and then validate
on many instances of the same class that the network had limited instances
to train/learn from. To ensure that each fold had comparable training and validation distributions 
for each class, we preprocessed the diary study data before sequentially partitioning it. 
Specifically, we divided the data of each diary study into its 
contiguous segments (e.g. 10:00-11:15, Table or 13:55-14:30, Hand), grouped 
those segments by class, and then concatenated the segments together for each class. The result was four 
homogenous datasets, one of each class, such that for each fold, we could construct
the validation set by aggregating together $\frac{1}{k}$ of each class dataset and then
construct the training set by compiling the remaining $\frac{k - 1}{k}$ of each class dataset.
Within these homogenous class datasets, the data were not randomized but 
kept sequential.

The average accuracy across all folds was $92.06\%$, and the corresponding
confusion matrix is shown in Figure X. The network is able to learn how to 
classify the Table state very accurately, which is understandable given 
the consistent nature of the state (i.e. always in a still, flat position). The states
of Backpack and Pocket were also classified with pretty moderate accuracy,
with the misclassifications likely stemming from the more diverse nature of 
the two states (i.e. the user could have been moving or still when the phone was
in their backpack or pocket). The Hand state was classified with the least accuracy,
and misclassified most often instead as the Backpack state. These misclassifications
likely also stem from the varied positions of the Hand state, as in addition to the user
using the phone actively in their Hand, they may also walk with their phone inactive
in their hand [@Joanna what kind of hand states did you have?]

 \begin{figure}[t]
 \center
\begin{tabular}{| l || c | c | c | c |}  
\toprule
\multicolumn{5}{c}{\textit{Actual Class}} \\ \cmidrule{2-5}
\textit{Predicted Class}		&	Backpack    & 	Pocket 	& 	Hand	&	Table \\
\midrule
Backpack			&	0.266 	&	0.005	&	0.024 	&	0.029 \\
Pocket			&	0.016 	&	0.041 	&	0.001 	&	0.001 \\
Hand			&	0.017 	&	0.003 	&	0.035 	&	0.006 \\
Table			&	0.004 	&	0.002 	&	0.002 	&	0.965\\
\bottomrule
\end{tabular}

\caption{Confusion matrix of the network predictions. Each entry indicates the proportion of
total instances that were predicted as the predicted class by the network and labeled the actual class.}
\centering
\label{fig:confusion}
\end{figure}  


\subsection{Multiple Phone Models}
We also attempted to validate our networks across different phone models, by training
on data collected by one phone model and then validating on data collected on a 
different phone model. 

However, the same network trained on the Nexus 5X data could unfortunately not be immediately
applied to the Nexus 5 data, as a cursory comparison of the accelerometer data between the two phones 
(while both phones had been on the table) revealed both differing accelerometer readings and readings
that did not equal the expected values of $X = 0g$, $Y = 0g$, and $Z = 9.8g$. Samples, 0.5s long, of 
the two phone's accelerometer data is shown in Figure Y. 

\begin{figure}[t]
\center
\includegraphics[scale=0.25]{won_table}
\includegraphics[scale=0.25]{joanna_table}
\caption{Graphs of the acceleration of the Nexus 5 and 5X while on a table (at different times).}
\end{figure}

In order to account for the inconsistent calibrations between the accelerometers of different phone models, we
attempted to recalibrate the accelerometer data for each phone before using it with the network. 
First, to see what type of recalibration model was necessary (e.g. $accel_{calibrated} = accel_{raw} + C$ or
$accel_{calibrated} = K(accel_{raw}) + C$), we taped the Nexus 5 and Nexus 5X together, and then recorded
the phones' accelerations in the four states. Plots of the two phones' accelerations against each other over the same time period
then suggested a linear relationship of the form $accel_{calibrated} = accel_{raw} + C$. 

\begin{figure}[t]
\center
\includegraphics[scale=0.5]{two_phones}
\caption{Graphs of the acceleration of the Nexus 5X (x1/y1/z1) against the Nexus 5 (x2/y2/z2) while taped together over the same period of time.}
\end{figure}

To find the calibration constants for each phone model, we collected accelerometer
data from when the phone was flat on a table, measured the acceleration in the
$X$, $Y$, and $Z$ directions and then computed the offsets from the expected values of
$0g$, $0g$, and $9.8g$. These offsets were then added to all of the accelerometer data for that phone model.

We applied this calibration process to both the Nexus 5X and Nexus 5 data, and then
trained a network on all of the Nexus 5X data, and validated individually on each of the 
two days of diary study data collected from the Nexus 5.
Nonetheless, even after calibration, results between the two phones were inconclusive. 
On the first diary study with the Nexus 5, our network
had a validation accuracy of only $76\%$, but on the second diary study, our network had a validation
accuracy of $91\%$--an accuracy much more comparable to the cross validation accuracy
demonstrated in the single phone model case. These results suggest that a linear calibration
strategy may be effective, but further investigation is needed for definitive understanding.

\section{Conclusion}
The work described in this paper describes the methodology in applying deep learning to the task of determining the state of a smartphone on the user's person. 
To do this, we do not require the user to be performing any specific action.
Instead, we utilize data from the accelerometer and screen, which are both lightweight and readily available on Android phone models, to identify four common phone positions that span most of a user's phone behavior
and appear distinguishable from the sensor data. 
We show that great accuracy can be achieved when evaluating on a single phone model.
Furthermore, we propose an accelerometer calibration strategy for standardizing phone accelerometer
data across phone models, and show the potential generalization of a network trained on data from a single phone model to other phone models.
 However, our cross model results are inconclusive. 
 For future work, we plan to strengthen these findings with the collection of more data, specifically the hand and pocket cases as well as investigate the classifier across phone models.
We would also like to experiment with the `smoothing' of outputs and determine if it can enhance accuracy.

\nocite{}

\bibliography{paper}
\bibliographystyle{apalike}


\end{document}
