\section{Evaluation}

\subsection{Data Collection Tools}
\subsubsection{Data Collection Software}
The software used to collect data for this experiment is the AppMon logging mobile application 
for the Android phone. 
The AppMon application logs the smartphone sensors, including the following:
\begin{enumerate}
\item Accelerometer
\item Number of unlocks
\item Number of screen touches
\item Number of times the screen turns on and off
\end{enumerate}

In our project, we will be using data from the sensors
to create features in order to classify and predict the location of the phone. 

The app collects accelerometer readings (X, Y, Z directions) in units of $m/s^2$ every 10ms.
It also registers the timestamp at which various trigger events occur (e.g. phone unlocks, screen on, etc.)

\subsubsection{Phones}
We used two phones for the purpose of diary studies and collecting data for creating the classifiers.
One phone was a Nexus 5 and the other was a Neuxs 5s


\subsection {Diary Study}
To obtain data for training and validation, one participant used the test phone with the app installed 
and underwent a daily routine.
Whenever a change of state occured [Note: As of this point, would the reader know what the states are?], the participant would record the time (according to the phone) and the new state.


The distribution of the states during the diary study are shown below:


\begin{center}
\begin{tabular}{ |c|c| } 
 \hline
 Table & 9.288 hours & 0.635\% \\ 
 Pocket & 0.492 hours & 0.034\%\\ 
 Backpack &  4.263 hours  &  0.292\% \\ 
 Hand & 0.576 hours  & 0.039\% \\
 \hline
\end{tabular}
\end{center}