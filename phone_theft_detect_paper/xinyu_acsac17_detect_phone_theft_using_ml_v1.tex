\documentclass[sigconf]{acmart}

\usepackage{booktabs} % For formal tables
\usepackage{multirow}


% Copyright
%\setcopyright{none}
%\setcopyright{acmcopyright}
%\setcopyright{acmlicensed}
\setcopyright{rightsretained}
%\setcopyright{usgov}
%\setcopyright{usgovmixed}
%\setcopyright{cagov}
%\setcopyright{cagovmixed}


% DOI
\acmDOI{10.475/123_4}

% ISBN
\acmISBN{123-4567-24-567/08/06}

%Conference
\acmConference[ACSAC'17]{Annual Computer Security Applications Conference}{December 2017}{San Juan, Puerto Rico, USA} 
\acmYear{2017}
\copyrightyear{2017}

\acmPrice{15.00}


\begin{document}
\title{Detecting Phone Theft Using Machine Learning}
% \titlenote{Produces the permission block, and
%   copyright information}
% \subtitle{Extended Abstract}
% \subtitlenote{The full version of the author's guide is available as
%   \texttt{acmart.pdf} document}

\author{Author Names Omitted for Anonymous Review. Paper-ID XXX}

\begin{abstract}
Every year, millions of smartphones in the United States are stolen, exposing victims' private information at risk since many users often do not lock their phones. To protect individuals' smartphones, we developed a system to automatically detect pick-pocket, and grab-and-run theft, where a thief grabs the phone from a victim's hand then runs away. We use binary classifiers to classify theft and normal usage with features extracted from accelerometer data. We collected a data set which consists of positive samples from simulated theft experiments and negative samples from a user study about normal usage of smartphones and built three models, among which logistic regression had the best performance. Logistic regression classifier detects 96.7\% of theft at a cost of 2 false alarms per day.
\end{abstract}

\keywords{Usable Security, Machine Learning, Phone-Theft Detection}

% \author{Ben Trovato}
% \authornote{Dr.~Trovato insisted his name be first.}
% \orcid{1234-5678-9012}
% \affiliation{%
%   \institution{Institute for Clarity in Documentation}
%   \streetaddress{P.O. Box 1212}
%   \city{Dublin} 
%   \state{Ohio} 
%   \postcode{43017-6221}
% }
% \email{trovato@corporation.com}

% \author{G.K.M. Tobin}
% \authornote{The secretary disavows any knowledge of this author's actions.}
% \affiliation{%
%   \institution{Institute for Clarity in Documentation}
%   \streetaddress{P.O. Box 1212}
%   \city{Dublin} 
%   \state{Ohio} 
%   \postcode{43017-6221}
% }
% \email{webmaster@marysville-ohio.com}

% \author{Lars Th{\o}rv{\"a}ld}
% \authornote{This author is the
%   one who did all the really hard work.}
% \affiliation{%
%   \institution{The Th{\o}rv{\"a}ld Group}
%   \streetaddress{1 Th{\o}rv{\"a}ld Circle}
%   \city{Hekla} 
%   \country{Iceland}}
% \email{larst@affiliation.org}

% \author{Lawrence P. Leipuner}
% \affiliation{
%   \institution{Brookhaven Laboratories}
%   \streetaddress{P.O. Box 5000}}
% \email{lleipuner@researchlabs.org}

% \author{Sean Fogarty}
% \affiliation{%
%   \institution{NASA Ames Research Center}
%   \city{Moffett Field}
%   \state{California} 
%   \postcode{94035}}
% \email{fogartys@amesres.org}

% \author{Charles Palmer}
% \affiliation{%
%   \institution{Palmer Research Laboratories}
%   \streetaddress{8600 Datapoint Drive}
%   \city{San Antonio}
%   \state{Texas} 
%   \postcode{78229}}
% \email{cpalmer@prl.com}

% \author{John Smith}
% \affiliation{\institution{The Th{\o}rv{\"a}ld Group}}
% \email{jsmith@affiliation.org}

% \author{Julius P.~Kumquat}
% \affiliation{\institution{The Kumquat Consortium}}
% \email{jpkumquat@consortium.net}

% % The default list of authors is too long for headers}
% \renewcommand{\shortauthors}{B. Trovato et al.}


% \begin{abstract}
% This paper provides a sample of a \LaTeX\ document which conforms,
% somewhat loosely, to the formatting guidelines for
% ACM SIG Proceedings.\footnote{This is an abstract footnote}
% \end{abstract}

%
% The code below should be generated by the tool at
% http://dl.acm.org/ccs.cfm
% Please copy and paste the code instead of the example below. 


\begin{CCSXML}
<ccs2012>
 <concept>
  <concept_id>10010520.10010553.10010562</concept_id>
  <concept_desc>Computer systems organization~Embedded systems</concept_desc>
  <concept_significance>500</concept_significance>
 </concept>
 <concept>
  <concept_id>10010520.10010575.10010755</concept_id>
  <concept_desc>Computer systems organization~Redundancy</concept_desc>
  <concept_significance>300</concept_significance>
 </concept>
 <concept>
  <concept_id>10010520.10010553.10010554</concept_id>
  <concept_desc>Computer systems organization~Robotics</concept_desc>
  <concept_significance>100</concept_significance>
 </concept>
 <concept>
  <concept_id>10003033.10003083.10003095</concept_id>
  <concept_desc>Networks~Network reliability</concept_desc>
  <concept_significance>100</concept_significance>
 </concept>
</ccs2012>  
\end{CCSXML}

\ccsdesc[500]{Computer systems organization~Embedded systems}
\ccsdesc[300]{Computer systems organization~Redundancy}
\ccsdesc{Computer systems organization~Robotics}
\ccsdesc[100]{Networks~Network reliability}


% \keywords{ACM proceedings, \LaTeX, text tagging}


\maketitle

\input{xinyu_acsac17_detect_phone_theft_using_ml_body_v1}

\bibliographystyle{ACM-Reference-Format}
\bibliography{sigproc} 

\end{document}
